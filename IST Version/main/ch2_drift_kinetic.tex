\chapter{A Drift-Kinetic Model for Scrape-off Layer Plasma Dynamics}
\label{ch:dk}

A physical theory describing the dynamics of magnetized plasma systems is considered to be closed and, more fundamentally, predictive, if it provides a constitutional relation for the sources of Maxwell's equations, namely the charge density $\rho$ and current density $\mathbf{J}$, in terms of the electromagnetic fields $(\phi, \mathbf{A})$.
%
Kinetic theory achieves this goal by providing a distribution function $f_a$ for each species $a$ in the plasma, where $f_a$ is a measure of the number of particles of species $a$ near point $\mathbf x$, having velocity $\mathbf v$, at time $t$ per unit volume and is normalized such that $\int f d\mathbf x d\mathbf v=N$ with $N$ the total number of particles in the system.
%
When $f_a$ is known, the charge density and current density can be obtained by taking velocity moments of $f_a$, namely with $\rho=\sum_a q_a \int f_a d\mathbf v$ and $\mathbf{J}= \sum_a q_a \int \mathbf v f_a d \mathbf v$ where $q_a$ is the charge of the species $a$.

The equation for the evolution of $f_a$ is derived from the analysis of the trajectories of the particles in the plasma.
%
When the details of particular temporal or spatial scales can be neglected, the equation for the evolution of $f_a$ can be greatly simplified.
%
This is the case of drift-kinetic theory, where the description of the charged particles inside the plasma is reduced to the behavior of its guiding-centers \citep{Hazeltine2003}.
%
This is particularly useful in the SOL, where fluctuations are characterized by frequencies lower than the ion gyrofrequency \citep{Endler1995,Agostini2011,Carralero2014,Garcia2015}, and the turbulent eddies, which include coherent radial propagation of filamentary structures \citep{DIppolito2002,DIppolito2011,Carreras2005,Serianni2007}, have a radial extension comparable to the time-averaged SOL pressure gradient length $L_p$ \citep{Zweben2007}.

{In recent years, there has been a significant development of first-principles simulations of the SOL dynamics} with both kinetic \citep{Tskhakaya2012} and gyrokinetic \citep{Xu2007, Shi2015, Chang2017,Shi2017} codes.
%
However, as kinetic simulations {of the SOL and edge regions} remain prohibitive as they still are computationally extremely expensive, the less demanding fluid simulations are the standard of reference.
%
The fluid simulations are usually based on the drift-reduced Braginskii \citep{Braginskii1965,Zeiler1997} or gyrofluid \citep{Ribeiro2008a,Held2016} models to evolve plasma density, fluid velocity and temperature. Fluid models assume that the distribution function is close to a local Maxwellian, and that scale lengths along the magnetic field are longer than the mean free path.
%
However, kinetic simulations show that the plasma distribution function is far from Maxwellian in the SOL region \citep{Tskhakaya2008,Lonnroth2006,Battaglia2014} and that collisionless effects in the SOL might become important \citep{Batishchev1997}.
%
This is expected to be particularly true in ITER and other future devices that will be operated in the high confinement mode (H-mode) regime \citep{Martin2008}.
%
In such cases, a transport barrier is formed that creates a steep pressure gradient at the plasma edge.
%
If the pressure gradient exceeds a threshold value, ELMs are destabilized \citep{Leonard2014}, expelling large amounts of heat and particles to the wall.
%
Describing structures with such high temperatures (and therefore low collisionality) with respect to the background SOL plasma requires therefore a model that allows for the treatment of arbitrary collision frequencies.
%
A kinetic full-F description is therefore needed for a proper SOL description \citep{Hazeltine1998}.

Leveraging the development of previous models {\citep{Hammett1993,Dorland1993,Beer1996,Sugama2001,Ji2010,Zocco2011,Schekochihin2016,Hatch2016,Parker2016,Hirvijoki2016,Mandell2018}}, we construct here a moment-hierarchy to evolve the SOL plasma dynamics.
%
Our model is valid in arbitrary magnetic field geometries and, making use of the full Coulomb collision operator, at arbitrary collision frequencies.
%
The model is derived within a full-F framework, as the amplitude of the background and fluctuating components of the plasma parameters in the SOL have comparable amplitude. We work within the drift approximation \citep{Hinton1976, Cary2009}, which assumes that plasma quantities have typical frequencies that are small compared to the ion gyrofrequency $\Omega_i=e B /m_i$, and their perpendicular spatial scale is small compared to the ion sound Larmor radius $\rho_s=c_s/\Omega_i$, with $c_s^2 = T_e/m_i$, $T_e$ the electron temperature, $B$ the magnitude of the magnetic field, $e$ the electron charge, and $m_i$ the ion mass.

In this chapter, we use the methods of Lagrangian mechanics to derive the equations of motion of a charged particle in an electromagnetic field in the drift-kinetic approximation, that is, when the magnetic field is slowly varying with respect to the gyroradius, and when fluctuations occur on spatial scales larger than the ion gyroradius.
%
A detailed description of the drift-kinetic ordering is provided in \cref{sec:ordering}.
%
In \cref{sec:solparticle}, we derive the drift-kinetic Lagrangian and state the equations governing the particle position and velocity in the drift-kinetic approximation, together with the equation for the evolution of the distribution function, the so-called drift-kinetic equation.
%
The drift-kinetic equation, when coupled to Maxwell's equations, yields a system of equations describing the dynamics of the plasma system that is, in principle, closed.
%
However, the numerical solution of kinetic models such as the drift-kinetic one still remains computationally extremely demanding.
%
For this reason, the drift-kinetic equation is converted into a moment-hierarchy equation for the evolution of the velocity moments of the distribution function $f_a$ using a suitable polynomial expansion of $f_a$ i.e., using a Hermite-Laguerre polynomial basis.
%
The expansion of the distribution function in a Hermite-Laguerre basis is performed in \cref{sec:momentexpansion}, while the moment-hierarchy equation is derived in \cref{sec:momenthierarchy}.
%
A shifted-velocity formulation, which retains the presence of a finite flow velocity in the Hermite-Laguerre basis and better captures strong near-Maxwellian flows with fewer expansion coefficients, is used.
%
A particular novelty of the framework derived here is the inclusion of collisions by evaluating explicitly the velocity moments of the full Coulomb nonlinear collision operator (the prefix full is used here to state that both like-particle and unlike-particle collisions are included).
%
This allows us to describe turbulent systems arbitrarily far from equilibrium using a model that is particularly efficient for numerical implementation.
%
In \cref{sec:poisson}, the system of equations is closed by deriving Poisson's equation in terms of coefficients of the Hermite-Laguerre expansion of the distribution function.
%
Finally, a a fluid model based on the truncation of the Hermite-Laguerre expansion in the high collisionality regime is presented, which allows the comparison to well-known fluid models used to describe the plasma dynamics in the SOL.
%
The conclusions follow.
%
We note that the results described in the present chapter have been published in \citet{Jorge2017}.

\section{Ordering}
\label{sec:ordering}

Denoting {$k_\perp \sim |\nabla_\perp \log \phi|\sim |\nabla_\perp \log n|\sim |\nabla_\perp \log T_{e}|$ and $\omega \sim |\partial_t \log \phi| \sim |\partial_t \log n| \sim |\partial_t \log T_e|$}, with $\phi$ the electrostatic potential, we introduce the drift-kinetic ordering parameter $\epsilon$ such that\footnote[1]
%
{\label{note1}We note that while this ordering differs from the one presented in \citet{Jorge2017}, the set of equations presented to describe SOL plasmas remains unchanged.
%
We also point out that the ordering $\omega \sim k_\parallel c_s$ may become marginal near separatrix where $k_\parallel$ decreases to values below than $\omega/c_s$.
}

\begin{equation}
    \epsilon \sim k_\perp \rho_s \sim \frac{k_\parallel}{k_\perp} \ll 1.
    \label{eq:ordering}
\end{equation}

\noindent {On the other hand, we let $k_\perp L_p \sim 1$ since turbulent eddies {are observed to have} an extension comparable to the scale lengths of the {time}-averaged quantities.}
These assumptions are in agreement with experimental measurements of SOL plasmas \citep{LaBombard2001,Zweben2004,Myra2013,Carralero2014}.
%
We set turbulence to be correlated along the magnetic field lines by ordering $\omega \sim k_\parallel c_s$ (see \cref{note1}), such that
%
\begin{equation}
    \frac{\omega}{\Omega_i} \sim \epsilon^2,
\label{eq:ordering2}
\end{equation}
%
an ordering in agreement with previous drift-reduced fluid models for the SOL \citep{Zeiler1997,Catto2004}.
%
We also order the electron collision frequency $\nu_{ei}$ as

\begin{equation}
	\frac{\nu_{ei}}{ \Omega_i} \sim \epsilon_\nu < \epsilon,
	\label{eq:orderingnu}
\end{equation}
%
In addition, the ion collision frequency ${\nu_i =} \nu_{ii}$ is ordered as $\nu_{ii} < \epsilon^2 \Omega_i$ that, noticing $\nu_i \sim  \sqrt{{m_e}/{m_i}}(T_e/T_i)^{3/2} \nu_e$ {(with $\nu_e = \nu_{ei}$)}, yields
\begin{equation}
    \left(\frac{\epsilon_\nu}{\epsilon^2}\right)^{2/3}\left(\frac{m_e}{m_i}\right)^{1/3}\lesssim\frac{T_i}{T_e}\lesssim 1.
    \label{eq:titebound}
\end{equation}
The ordering in \cref{eq:titebound} can be used to justify applying our model in the cold ion limit, $T_i \ll T_e$, {but allows for $T_i \sim T_e$}.
We note that in the SOL the ratio $T_i/T_e$ is typically in the range $1 < T_i/T_e < 4$ \citep{Kocan2011}. Furthermore, it is seen that the ion temperature in this range of values plays a negligible role in determining the SOL turbulent dynamics, usually due to a steeper electron temperature profile compared with the ion one, which is usually below the threshold limit of the ion temperature gradient instability \citep{Mosetto2015}.

The ordering in Eqs. (\ref{eq:ordering})-(\ref{eq:titebound}) is justified in a wide variety of experimental conditions. For example, for a typical JET discharge \citep{Erents2000,Liang2007,Xu2009} with the SOL parameters $B_T = 2.5$ T, $T_e \sim T_i \sim 20$ eV, $n_e \simeq 10^{19}$ m$^{-3}$, and $k_\perp \sim 1$ cm$^{-1}$, we obtain $\epsilon_\nu \sim 0.016$ and $\epsilon \sim 0.0182$. For a medium-size tokamak such as TCV \citep{Rossel2012,Nespoli2017}, estimating  $B_T = 1.5$ T, $T_e \sim T_i \sim 40$ eV, $n_e \simeq 6 \times 10^{18}$, and $k_\perp \sim 1$ cm$^{-1}$, we obtain $\epsilon_\nu \sim 6.2 \times 10^{-3}$ and $\epsilon \sim 0.043$. Finally, for small-size tokamaks such as ISTTOK \citep{Silva2011a,Jorge2016}, with $B_T = 0.5$ T, $T_e \sim T_i \sim 20$ eV, $n_e \simeq 0.8 \times 10^{18}$, and $k_\perp \sim 1$ cm$^{-1}$, we obtain $\epsilon_\nu \sim 0.0072$ and $\epsilon \sim 0.091$. Lower values of $\epsilon_\nu$, as in the presence of ELMs where temperatures can reach up to $100$ eV \citep{Pitts2003}, are also included in the ordering considered here.
%
%Following typical SOL experimental measurements (see, e.g. \citet{Zweben2007,Terry2009,Grulke2014}), we order $k_{\parallel} \sim 1/L_B \sim 1/R$, with $L_B$ the background magnetic field spatial gradient scale and $R$ the tokamak major radius, and take $k_\parallel \rho_s \sim \epsilon^3$. This yields
%
%\begin{equation}
    %\frac{k_\parallel}{k_\perp} \sim \epsilon^2,
    %\label{eq:kpar1}
%\end{equation}
%
%\noindent a lower ratio than the ones used in most drift-kinetic and gyrokinetic deductions \citep{Hahm1988a,Hazeltine2003,Abel2013}.
%
We note that the orderings in \cref{eq:orderingnu,eq:ordering,eq:ordering2} imply that

\begin{equation}
    k_\parallel \lambda_{mfp} \sim \sqrt{\frac{m_i}{m_e}}\frac{\epsilon^2}{\epsilon_\nu},
    \label{eq:kparlmfp}
\end{equation}
%
which includes both the collisional regime $k_{\parallel} \lambda_{mfp} \ll 1$, when $\epsilon_\nu \sim \epsilon$, and the collisionless regime $(k_{\parallel} \lambda_{mfp})^{-1} \ll 1$, when $\epsilon_\nu \ll \epsilon$.
%
Finally, the plasma parameter $\beta = n T_e/(B^2/2\mu_0)$ is ordered as $\beta \sim \epsilon^3$, {implying that our equations describe plasma dynamics in an electrostatic regime}.
%
Although electromagnetic effects can lead to a non-negligible enhancement on heat and particle transport in the SOL \citep{LaBombard2005}, we focus on devices with low-enough $\beta$ such that the value of the MHD ballooning parameter $\alpha_{\text{MHD}} = \beta R/L_p$ stays below the electromagnetic balloning instability threshold.
%
We refer the reader to \citet{Halpern2013a} for a detailed treatment of electromagnetic effects in the SOL within the drift-reduced fluid description and here we consider the electrostatic limit.

\section{SOL Guiding-Center Model}
\label{sec:solparticle}

\subsection{Single-Particle Motion}

To derive a convenient equation of motion in the presence of a strong magnetic field $\mathbf B$, we start with the Hamiltonian of a charged particle of species $a$ \citep{Jackson1999},
%
\be
    H_a(\mathbf q, \mathbf p)=\frac{[\mathbf p - q_a \mathbf A(\mathbf q)]^2}{2m_a}+q_a \phi(\mathbf q),
    \label{eq:hamiltonian}
\ee
%
and its associated Lagrangian,
%
\begin{equation}
    L_a(\mathbf x, \mathbf v)=\left[q_a \mathbf A(\mathbf x) + m_a \mathbf v\right]\cdot \dot{\mathbf x}-\left[\frac{m_a v^2}{2}+q_a \phi(\mathbf x)\right],
    \label{eq:lagrangian}
\end{equation}
%
where $\mathbf p = q_a \mathbf A + m _a\mathbf v$ is the canonical momentum conjugated to $\mathbf q = \mathbf x$, $\mathbf v$ is the particle velocity, $\mathbf A$ is the magnetic vector potential, $\phi$ is the electrostatic potential, $m_a$ is the mass of the particle and $q_a$ its charge.

We now perform a coordinate transformation from the phase-space coordinates $\mathbf z = (\mathbf x, \mathbf v)$ to the guiding-center coordinates $\mathbf Z = (\mathbf R, v_\parallel, \mu, \theta)$ by writing the particle velocity as [see, e.g., \citet{Littlejohn1983a}]
%
\begin{align}
    \mathbf v &= \mathbf U+ v_\perp' \mathbf c,
    \label{eq:GCcoordinates}
\end{align}
%
with
%
\begin{equation}
    \mathbf U = \mathbf v_E(\mathbf R) + v_\parallel \mathbf b(\mathbf R),
\end{equation}
%
and $\mathbf v_E = \mathbf E \times \mathbf B/B^2$ the $\mathbf E \times \mathbf B$ velocity. The gyroangle $\theta$, defined as
%
\begin{equation}
    \theta = \tan^{-1} \left[\frac{\left(\mathbf v - \mathbf U \right)\cdot \mathbf e_2}{\left(\mathbf v - \mathbf U \right)\cdot \mathbf e_1} \right]
\end{equation}
%
is introduced by defining the right-handed coordinate set $(\mathbf e_1, \mathbf e_2, \mathbf b)$, such that $\mathbf c = -\mathbf a \times \mathbf b=\mathbf d a(\theta)/d\theta$, with $\mathbf a = \cos \theta \mathbf e_1 + \sin \theta \mathbf e_2$.
The decomposition in \cref{eq:GCcoordinates} allows us to isolate the high-frequency gyromotion contained in the $v_\perp' \mathbf c$ term, from the dominant guiding-center velocity $\mathbf U$.
The adiabatic invariant $\mu$ is defined as
%
\begin{equation}
    \mu = \frac{m_a v_\perp^{'2}}{2B}
\label{eq:gcmu}
\end{equation}
%
whereas the guiding-center position is
%
\begin{equation}
    \mathbf R = \mathbf x - \rho_a \mathbf a,
    \label{eq:GCx}
\end{equation}
%
with $\rho_a = \sqrt{2 m_a  \mu/(q_a^2 B)}$ the Larmor radius. Incidentally, for the case of weakly varying magnetic fields, \cref{eq:GCx} describes the circular motion of a particle around its guiding-center $\mathbf R$ with radius $\rho_a$, i.e., $(\mathbf x - \mathbf R)^2 = \rho_a^2$.

As our goal is to develop a model that describes turbulent fluctuations occurring on a spatial scale longer than the sound Larmor radius $\rho_s$, and a time scale larger than the gyromotion one, we keep terms in the Lagrangian up to $O(\epsilon)$ and order $T_i \sim T_e$, which implies
%
\begin{equation}
    k_{\perp} \rho_i \sim \epsilon.
    \label{eq:tiordering}
\end{equation}
%
We therefore expand the electromagnetic fields around $\mathbf R$, to first order in $\epsilon$, i.e.,
%
\begin{equation}
    \phi(\mathbf x) \simeq \phi(\mathbf R) + \rho_a \mathbf a \cdot \nabla_{\mathbf R} \phi(\mathbf R),
    \label{eq:FLRexp}
\end{equation}
%
and similarly for $\mathbf A$. In the following, if not specified, the electromagnetic fields and potentials are evaluated at the guiding-center position $\mathbf R$, and we denote $\nabla=\nabla_{\mathbf R}$.
In addition, to take advantage of the difference between the turbulent and gyromotion time scales, we use the gyroaveraged Lagrangian $\lb L_a \rb_{\mathbf R}$ to evaluate the plasma particle motion, where the gyroaveraging operator $\lb \chi \rb_{\mathbf R}$ acting on a quantity $\chi(\theta)$ is defined as
%
\begin{equation}
    \lb \chi \rb_{\mathbf R} = \frac{1}{2\pi}\int_0^{2\pi} \chi (\theta) d\theta,
\label{gyaveroperator}
\end{equation}
%
which is performed at fixed guiding-center coordinates $\mathbf R$, $v_\parallel$ and $\mu$.

To evaluate $\lb L_a \rb_{\mathbf R}$ we note that, with the expansion for $\phi$ and $\mathbf A$, the Lagrangian in \cref{eq:lagrangian} can be expressed as $L_a=L_{0a}+L_{1a}+\tilde L_a$ where $L_{0a}$ is gyroangle independent,
%
\begin{equation}
    \begin{split}
        L_{0a} &= \left(q_a \mathbf A + m_a \mathbf U\right)\cdot \dot{\mathbf R}-\left(\frac{m_a v_\parallel^2}{2}+\frac{m_a v_E^2}{2}+\mu B+q_a \phi\right),
    \end{split}
    \label{eq:L0}
\end{equation}
%
$L_{1a}$ is proportional to $\rho_a^2$ (and hence to $\mu$) and is order $\epsilon^0$
%
\begin{equation}
    \begin{split}
        &L_{1a} = \rho_a^2 q_a \dot \theta \left(\mathbf a \cdot \nabla\right) \left(\mathbf A \cdot \mathbf c\right)+m_a \rho_a^2 \Omega \dot \theta+{\rho_a \dot \rho_a}\left[q_a \left(\mathbf a \cdot \nabla\right)\left(\mathbf A \cdot \mathbf a\right)\right],
    \end{split}
    \label{eq:L2}
\end{equation}
%
and the $\tilde L_{a}$ contribution contains the terms {linearly proportional to $\cos \theta$ or $\sin \theta$} \citep{Cary2009} {which are not present in $\lb L_a \rb_{\mathbf R}$, as $\lb \tilde L_a \rb_{\mathbf R} = 0$}.

We note that $\lb L_{1a} \rb_{\mathbf R}$ can be simplified since $\lb \left(\mathbf a \cdot \nabla\right) \mathbf A \cdot \mathbf c \rb_{\mathbf R} = -\mathbf b \cdot (\nabla \times \mathbf A) /2$, and $\lb \left(\mathbf a \cdot \nabla\right) \mathbf A \cdot \mathbf a \rb_{\mathbf R} = \nabla_\perp\cdot\mathbf{A}/2$. Subtracting the total derivative $-q_a d/dt(\rho_a^2 \nabla_\perp \mathbf A)/4$ from $\lb L_a \rb_{\mathbf R}$, which does not alter the resulting equations of motion, we redefine the gyroaveraged Lagrangian as
%
\begin{equation}
    \begin{split}
        \lb L_a \rb_{\mathbf R} &= \left(q_a\mathbf A + m_a{\mathbf U}\right) \cdot \dot{\mathbf R}  - \left(\frac{m_a v_\parallel^2}{2}+\frac{m_a v_E^2}{2} + q_a \phi\right)-\mu B\left(1- \frac{\dot \theta}{\Omega_a}\right)
        -\frac{\rho_a^2}{4}\frac{d }{dt}\left[\nabla_\perp\cdot\left(q_a \mathbf A\right)\right].
    \end{split}
    \label{eq:gyroLag}
\end{equation}
%
We now order the terms appearing in $\lb L_a \rb_{\mathbf R}$.
As imposed by the Bohm sheath conditions \citep{Stangeby2000}, both electrons and ions stream along the field lines with parallel velocities comparable to the sound speed $c_s = \sqrt{T_e/m_i}$ in the SOL.
The Bohm boundary conditions at the sheath also set the electrostatic potential $e \phi \sim \Lambda T_e$ across the SOL, where $\Lambda = \ln \sqrt{m_i/(m_e 2\pi)}\simeq 3$.
Therefore, we keep the $m_a v_E^2/2$ term in the Lagrangian in \cref{eq:gyroLag}, as to take into account the presence of the numerically large factor $\Lambda^2$ in $v_E^2 \sim \epsilon^2 \Lambda^2 c_s^2$.

By neglecting the higher-order terms in \cref{eq:gyroLag}, i.e., $-(\rho_a^2/4){d }\left[\nabla_\perp\cdot\left(q_a \mathbf A\right)\right]/{dt}$, the expression for the gyroaveraged Lagrangian describing SOL single-particle dynamics, up to $O(\epsilon)$, can be written as
%
\begin{equation}
    \lb L_a \rb_{\mathbf R} = q_a \mathbf A^* \cdot \dot{\mathbf R} - q_a \phi^* -\frac{m_a v_\parallel^2}{2}+ \mu \frac{m_a \dot \theta}{q_a}.
    \label{eq:lagSOL}
\end{equation}
%
where
%
\begin{equation}
    q_a\phi^* = q_a\phi+ m_a v_E^2/2+\mu B
\label{eq:phis}
\end{equation}
%
and
%
\begin{equation}
    q_a\mathbf A^* = q_a \mathbf A  + m_a v_\parallel \mathbf b + m_a \mathbf v_E.
\label{eq:As}
\end{equation}
%
The Euler-Lagrange equations applied to the Lagrangian in \cref{eq:lagSOL} for the coordinates $\theta$, $v_\parallel$, and $\mu$, yield, respectively, $\dot \mu = 0$, $v_\parallel = \mathbf b \cdot \dot{\mathbf R}$, and $\dot \theta = \Omega_a$. For the ${\mathbf R}$ coordinate, we obtain
%
\begin{equation}
    m_a \dot v_\parallel \mathbf b = q_a (\mathbf E^* + \dot{\mathbf R}\times \mathbf B^*),
    \label{eq:motLag}
\end{equation}
%
where the relation $[\nabla \mathbf A - (\nabla \mathbf A)^T] \cdot \dot{\mathbf R} = \dot{\mathbf R}\times (\nabla \times \mathbf A)$ has been used, and we defined $\mathbf E^* = -\nabla \phi^* - \partial_t \mathbf A^*$, and $\mathbf B^* = \nabla \times \mathbf A^*$, with the parallel component of $\mathbf B^{*}$ given by
%
\begin{equation}
    B_{\parallel}^* = \mathbf B^* \cdot \mathbf b = B + \frac{m_a}{q_a}\mathbf b \cdot \nabla \times \left(v_\parallel \mathbf b + \mathbf v_E\right).
\label{eq:defbpars}    
\end{equation}
%
By projecting \cref{eq:motLag} along $\mathbf B^*$, we derive $m \dot v_\parallel B_\parallel^*= e \mathbf E^* \cdot \mathbf B^*$, while crossing with $\mathbf b$ yields the guiding-center velocity $\dot{\mathbf R} B_\parallel^* = v_\parallel \mathbf B^* + \mathbf E^* \times \mathbf B/B$. Using the expressions for the fields $\mathbf E^{*}$ and $\mathbf B^{*}$, we obtain 
%
\be
    \dot{\mathbf R}  = \mathbf U+\frac{\mathbf B}{\Omega_a B_\parallel^*}\times\left(\frac{d \mathbf U}{dt}+\frac{\mu\nabla B}{m_a}\right),
    \label{eq:GC1}
\ee
%
and
%
\be
    m_a \dot v_\parallel = q_a E_\parallel - \mu \nabla_\parallel B + m_a \mathbf v_E \cdot \frac{d \mathbf b}{dt}-m_a\mathcal{A},
    \label{eq:GC2}
\ee
%
In \cref{eq:GC1,eq:GC2}, {in addition to the time derivatives of the phase-space coordinates $\dot{\mathbf R}, \dot v_{\parallel}$, that only have an explicit time dependence, we define the total derivative $d/dt$ of a field $\phi(\mathbf R, t)$ that has an explicit time and $\mathbf R$ dependence as
%
\begin{equation}
    \frac{d\phi}{dt} = \frac{\partial \phi}{\partial t} + \mathbf U \cdot \nabla \phi.
\end{equation}
%
The $\mathcal{A}$ term represents the higher-order nonlinear terms in $\dot v_\parallel$ that ensure phase-space conservation properties \citep{Cary2009}, and it is given by
%
\begin{equation}
    \mathcal{A}=\frac{B}{B_\parallel^*}\left(\left.\frac{d \mathbf U}{dt}\right|_\perp + \mu \nabla_\perp B\right)\cdot \frac{\nabla \times \mathbf U}{\Omega_a},
\end{equation}
%
with $d_t\mathbf U|_\perp=-\mathbf b \times (\mathbf b \times d_t\mathbf U)$.

The guiding-center equations of motion (\ref{eq:GC1}) and (\ref{eq:GC2}) satisfy the energy, $E_{gc} = q_a \phi^*+m_a v_\parallel^2/2$ \citep{Cary2009}, and momentum, $\mathbf P_{gc} = e \mathbf A^*$ \citep{Cary2009}, conservation laws, given by
%
\begin{equation}
    \frac{d E_{gc}}{dt} = q_a \frac{\partial \phi^*}{\partial t}- q_a \frac{\partial \mathbf A^*}{\partial t}\cdot \dot{\mathbf R},
    \label{eq:enconservation}
\end{equation}
%
and
%
\begin{equation}
    \frac{\partial \mathbf P_{gc}}{\partial t} = -q_a \nabla \phi^* + q_a \nabla \mathbf A^* \cdot \dot{\mathbf R}.
    \label{eq:moconservation}
\end{equation}
%
In addition, we note that using \cref{eq:GC1,eq:GC2} and Maxwell's equations, a conservation equation for $B_{\parallel}^*$ can be derived
%
\begin{equation}
    \frac{\partial B_{\parallel}^*}{\partial t} + \nabla \cdot (\dot{\mathbf R} B_{\parallel}^*) + \frac{\partial}{\partial v_\parallel}\left(\dot v_\parallel B_{\parallel}^* \right)=0.
    \label{eq:liouvilleGC}
\end{equation}
%
Since $B_{\parallel}^*$ is the Jacobian of the guiding-center transformation, \cref{eq:liouvilleGC} is in fact the phase-space volume conservation law for the guiding-center system of equations (also called Liouville's theorem), reflecting therefore their Hamiltonian nature.

\subsection{The Guiding-Center Boltzmann Equation}
\label{subsec:gcboltzmann}

The Boltzmann equation for the evolution of the distribution function $f_a(\mathbf x, \mathbf v)$ of the particles in $(\mathbf x, \mathbf v)$ coordinates is
%
\be
     \frac{\partial f_a}{\partial t}+\dot{\mathbf x}\cdot \nabla_{\mathbf x} f_a + \dot{\mathbf v}\cdot \nabla_{\mathbf v} f_a = C(f_a),
     \label{eq:boltzmann}
\ee
%
where $C(f_a)=\sum_b C(f_a,f_b) = \sum_b C_{ab}$ is the collision operator.
%
Because $f_a$ can significantly deviate from a Maxwellian distribution function in the SOL \citep{Battaglia2014}, we consider the {bilinear} Coulomb operator $C_{ab}$ \citep{Balescu1988}, to model collisions between particles of species $a$ and $b$
%
\be
    \begin{split}
        C_{ab}&=L_{ab} \frac{\partial}{\partial v_i}\left[\frac{\partial^2 G_b}{\partial v_i \partial v_j}\frac{\partial f_a}{\partial v_j}-\frac{m_a}{m_b}\frac{\partial H_b}{\partial v_i}f_a\right],
    \end{split}
    \label{eq:coulombop}
\ee
%
with
%
\be
    \begin{split}
        H_b&=2\int \frac{f_b(\mathbf v')}{|\mathbf v - \mathbf v'|}d\mathbf v',
    \end{split}
\label{eq:roshb}
\ee
%
and
%
\be
    \begin{split}
        G_b=\int f_b(\mathbf v')|\mathbf v - \mathbf v'|d\mathbf v',
    \end{split}
\ee
%
the Rosenbluth potentials satisfying $\nabla^2_v G_b = H_b$. In \cref{eq:coulombop} we introduced $L_{ab}=q_a^2 q_b^2 \lambda/(4 \pi \epsilon_0^2 m_a^2)=\nu_{ab} v_{tha}^3/n_b$, where $\lambda$ is the Coulomb logarithm, $\nu_{ab}$ the collision frequency between species $a$ and $b$, and $v_{tha}^2=2 T_a/m_a$.

Taking advantage of the small electron to ion mass ratio, the collision operator between unlike-species can be simplified [see, e.g. \citet{Balescu1988,Helander2002}]. 
The electron-ion collision operator, to first order in $m_e/m_i$, is given by the operator $C_{ei}(f_e)=C_{ei}^0+C_{ei}^1$, where $C_{ei}^0$ is the Lorentz pitch-angle scattering operator
\be
    \begin{split}
        C_{ei}^0&=\frac{n_i L_{ei}}{v_{the}^3}\frac{\partial}{\partial \mathbf c_e}\cdot\left[\frac{1}{c_e}\frac{\partial f_e}{\partial \mathbf c_e}-\frac{\mathbf c_e}{c_e^3}\left(\mathbf c_e \cdot \frac{\partial f_e}{\partial \mathbf c_e}\right)\right],
    \end{split}
    \label{eq:cei0}
\ee
%
and $C_{ei}^1$ the momentum-conserving term
%
\be
    \begin{split}
        C_{ei}^1&={\frac{2 n_i L_{ei}}{v_{the}^4 c_e^3}f_{Me}{\mathbf u_{i}} \cdot \mathbf c_e}.
    \end{split}
    \label{eq:cei1}
\ee
%
with $\mathbf c_a = (\mathbf v - \mathbf u_a)/v_{tha}$.
%
Ion-electron collisions, to first order in $m_e/m_i$, are desribed using the operator
%
\begin{equation}
    \begin{split}
        C_{ie}&=\frac{ \mathbf R_{ei}}{m_i n_i v_{thi}}\cdot \frac{\partial f_i}{\partial \mathbf c_i}%,\\
        %C_{ie}^1&=
        +\nu_{ei}\frac{n_e}{n_i}\frac{m_e}{m_i}\frac{\partial}{\partial \mathbf c_i}\cdot\left(\mathbf c_i f_i%,\\
        %C_{ie}^2&=\nu_{ei}\frac{m_e}{m_i}%
        +\frac{T_e}{T_i}
        %\frac{T_e}{m_i}
        \frac{\partial f_i}{\partial \mathbf c_i} 
        \right),
    \end{split}
    \label{eq:cie}
\end{equation}
%
where $\mathbf R_{ei}=\int m_e \mathbf v C_{ei} d\mathbf v$ is the electron-ion friction force.
%
We take advantage of \cref{eq:orderingnu} to order the electron collision frequency $\nu_e$ and the ion collision frequency $\nu_i$ as
%
\begin{equation}
    \frac{\nu_i}{\Omega_i} \sim \sqrt{\frac{m_e}{m_i}}\left(\frac{T_e}{T_i}\right)^{3/2}\epsilon_{\nu} < \epsilon^2,
    \label{eq:orderingnu2}
\end{equation}
%
where we used the relation $\nu_i \sim  \sqrt{{m_e}/{m_i}}(T_e/T_i)^{3/2} \nu_e$. The orderings in \cref{eq:orderingnu2,eq:FLRexp} yield the lower bound in \cref{eq:titebound} for the ion to electron temperature ratio.

We now express the particle distribution function $f_a$ in terms of the guiding-center coordinates {by defining $F_a$, a function of guiding-center coordinates, as}
%
\begin{equation}
    F_a(\mathbf R, v_\parallel, \mu, \theta) = f_a(\mathbf x(\mathbf R, v_\parallel, \mu, \theta), \mathbf v(\mathbf R, v_\parallel, \mu, \theta)).
    \label{eq:fguidF}
\end{equation}
%
Using the chain rule to rewrite \cref{eq:boltzmann} in guiding-center coordinates, we obtain
%
\be
     \frac{\partial F_a}{\partial t}+\dot{\mathbf R}\cdot \nabla F_a + \dot{v_\parallel}\frac{\partial F_a}{\partial v_\parallel} + \dot \mu \frac{\partial F_a}{\partial \mu}+ \dot \theta \frac{\partial F_a}{\partial \theta} = C(F_a),
     \label{eq:boltzmannSS}
\ee
%
where $\dot{\mathbf R}$ and $\dot v_\parallel$ are given by \cref{eq:GC1} and \cref{eq:GC2} respectively, $\dot \theta = \Omega_a$, and $\dot \mu = 0$.
Equation (\ref{eq:boltzmannSS}) can be simplified by applying the gyroaveraging operator in \cref{gyaveroperator}. This results in the drift-kinetic equation
%
\begin{equation}
    \frac{\partial \lb F_a \rb_{\mathbf R}}{\partial t}+ \dot{\mathbf R} \cdot \nabla\lb F_a \rb_{\mathbf R} + \dot v_{\parallel}\frac{\partial \lb F_a \rb_{\mathbf R}}{\partial v_\parallel} = \lb C(F_a)\rb_{\mathbf R}.
    \label{eq:boltzmannGC1}
\end{equation}
%
We now write \cref{eq:boltzmannGC1} in a form useful to take gyrofluid moments of the form $\int \lb F_a \rb_{\mathbf R} B dv_\parallel d\mu d\theta$ (see \cref{sec:momenthierarchy}). Using the conservation law in \cref{eq:liouvilleGC} for $B_{\parallel}^{*}$, we can write the guiding-center Boltzmann equation in conservative form as
%
\begin{equation}
    \begin{split}
        &\frac{\partial (B_{\parallel}^*\lb F_a \rb_{\mathbf R})}{\partial t}+ \nabla \cdot ( \dot{\mathbf R} B_{\parallel}^*\lb F_a \rb_{\mathbf R}) + \frac{\partial( \dot v_{\parallel a} B_{\parallel}^*\lb F_a \rb_{\mathbf R})}{\partial v_\parallel} = B_{\parallel}^*\lb C(F_a)\rb_{\mathbf R}. 
    \end{split}
    \label{eq:boltzmannGC}
\end{equation}
%
Moreover, in order to relate the gyrofluid moments $\int \lb F_a \rb_{\mathbf R} B dv_\parallel d\mu d\theta$ with the usual fluid moments $\int f_a d^3 v$, we estimate the order of magnitude of the gyrophase dependent part of the distribution function $\tilde F_a = F_a - \lb F_a \rb_{\mathbf R}$ where $\lb F_a \rb_{\mathbf R}$ obeys \cref{eq:boltzmannGC1}. The equation for the evolution of $\tilde F_a$ is obtained by subtracting \cref{eq:boltzmannGC1} from the Boltzmann equation, \cref{eq:boltzmannSS}, that is
%
\begin{equation}
    \frac{\partial \tilde F_a}{\partial t}+\dot{\mathbf R}\cdot \nabla \tilde F_a + \dot{v_\parallel}\frac{\partial \tilde F_a}{\partial v_\parallel} + \Omega_a \frac{\partial \tilde F_a}{\partial \theta} = C(F_a)-\lb C(F_a)\rb_{\mathbf R}.
    \label{eq:boltztilde}
\end{equation}

Using the orderings in \cref{eq:orderingnu,eq:orderingnu2}, as well as $\partial_t \sim \dot{\mathbf R} \cdot \nabla \sim \dot v_\parallel \partial_{v_\parallel} \sim \epsilon \Omega_i$ and $
\partial_\theta \sim 1$, the comparison of the leading-order term on the left-hand side of \cref{eq:boltztilde} with the right-hand side of the same equation imply the following ordering for $\tilde F_e$
%
\begin{equation}
    \frac{\tilde F_e}{\lb F_e \rb_{\mathbf R}} \sim \frac{m_e}{m_i}\epsilon_\nu< \epsilon^2,
    \label{eq:orderingftildee}
\end{equation}
%
and $\tilde F_i$
%
\begin{equation}
    \frac{\tilde F_i}{\lb F_i \rb_{\mathbf R}} \sim \sqrt{\frac{m_e}{m_i}}\left(\frac{T_e}{T_i}\right)^{3/2}\epsilon_\nu < \epsilon^2.
    \label{eq:orderingftildei}
\end{equation}
%
To evaluate the leading-order term of $\tilde F_a$, we expand the collision operator $C(F_a) = C_0(\lb F_a\rb_{\mathbf R}) + \epsilon C_1(F_a) + ...$, such that
%
\begin{equation}
    \tilde F_a \simeq \frac{1}{\Omega_a}\int_0^\theta\left[C_0(\lb F_a \rb_{\mathbf R} )-\lb C_0( \lb F_a \rb_{\mathbf R})\rb_{\mathbf R}\right]d\theta' + O( \epsilon^3 \lb F_a \rb_{\mathbf R}).
    \label{eq:tildefapp}
\end{equation}
%
The relation in \cref{eq:tildefapp} can be further simplified by expanding the $\theta$ dependence of $F_a$ in Fourier harmonics, 
%
\begin{equation}
    F_a=\sum_m e^{i m \theta} F_{m a},
    \label{eq:fourftilde}
\end{equation}
%
so that for $m=0$ we have $\lb F_a \rb_{\mathbf R} = F_{0a}$, and similarly for $C_0(\lb F_a \rb_{\mathbf R})$
%
\begin{equation}
    C_0(\lb F_a \rb_{\mathbf R}) = \sum_{m'} e^{i m' \theta} C_{m' a}.
\end{equation}
%
We can then write \cref{eq:tildefapp} as
%
\begin{equation}
    \tilde F_{m a} = \frac{C_{m a}}{i m \Omega_a},
    \label{eq:fmacmafourier}
\end{equation}
%
for $m \not=0$.

\section{Moment Expansion}
\label{sec:momentexpansion}

We now derive a polynomial expansion for the distribution function $\lb F_a \rb_{\mathbf R}$ that simplifies the solution of \cref{eq:boltzmannGC}, with the collision operators in Eqs. (\ref{eq:coulombop}) - (\ref{eq:cie}).
This section is organized as follows.
In \cref{section:gcmoment} the Hermite-Laguerre basis is introduced, relating the corresponding expansion coefficients for $\lb F_a \rb_{\mathbf R}$ with its usual gyrofluid moments.
In \cref{section:jifluidexpansion}, we briefly review the fluid moment expansion of the Coulomb collision operator presented in \citet{Ji2006, Ji2008}.
In \cref{section:cabmomentexpansion}, leveraging the work in  \citet{Ji2006, Ji2008}, we expand $C_{ab}$ in terms of the product of the gyrofluid moments, for both like- and unlike-species collisions which, ultimately, allows us to solve \cref{eq:boltzmannGC} in terms of gyrofluid moments.

\subsection{Guiding-Center Moment Expansion of \texorpdfstring{$\lb F_a \rb_{\mathbf R}$}{}}
\label{section:gcmoment}

To take advantage of the anisotropy introduced by a strong magnetic field, and efficiently treat the left-hand side of \cref{eq:boltzmannGC} where the parallel and perpendicular directions appear decoupled, we express $\lb F_a \rb_{\mathbf R}$ by using {a Hermite polynomial basis expansion for the parallel velocity coordinate \citep{Grad1949,Armstrong1967a,Grant1967,Ng1999,Zocco2011,Loureiro2013a,Parker2015,Schekochihin2016,Tassi2016} and a Laguerre polynomial basis for the perpendicular velocity coordinate {\citep{Zocco2015,Omotani2015,Mandell2018}}. More precisely, we use the following expansion
%
\be
    \begin{split}
        \lb F_a \rb_{\mathbf R} &=\sum_{p,j=0}^{\infty} \frac{N_a^{pj}}{\sqrt{2^p p!}}F_{Ma}  H_p(s_{\parallel a})L_j(s_{\perp a}^2),
    \end{split}
    \label{eq:gyrof}
\ee
%
where the {\textit{physicists'}} Hermite polynomials $H_p$ of order $p$ are defined by the Rodrigues' formula \citep{Abramowitz1972}
%
\begin{equation}
    H_p(x)=(-1)^p e^{x^2}\frac{d^p}{dx^p}e^{-x^2},
\end{equation}
%
and normalized via
%
\begin{equation}
    \int_{-\infty}^{\infty} dx H_p(x) H_{p'}(x) e^{-x^2} = 2^p p! \sqrt{\pi} \delta_{p{p'}},
\end{equation}
%
and the Laguerre polynomials  $L_j$ of order $j$ are defined by the Rodrigues' formula \citep{Abramowitz1972} 
%
\begin{equation}
    L_j(x)=\frac{e^x}{j!}\frac{d^j}{dx^j}(e^{-x}x^j),
\end{equation}
%
which are orthonormal with respect to the weight $e^{-x}$
%
\begin{equation}
    \int_{0}^{\infty} dx L_j(x) L_{j'}(x) e^{-x} = \delta_{jj'}.
\end{equation}
%
Because of the orthogonality of the Hermite-Laguerre basis, the coefficients $N_a^{pj}$ of the expansion in \cref{eq:gyrof} are
%
\be
    N_a^{pj}=\frac{1}{N_{a}}\int \frac{H_p(s_{\parallel a}) L_j(s_{\perp a}^2) \lb F_a \rb_{\mathbf R} }{\sqrt{2^p p!}}\frac{B}{m_a} d\mu dv_\parallel d\theta,
    \label{eq:gyromoments}
\ee
%
and correspond to the guiding-center moments of $\lb F_a \rb_{\mathbf R}$.

In \cref{eq:gyrof}, the shifted bi-Maxwellian is introduced
%
\be
	F_{Ma}=N_a\frac{e^{-{s_{\parallel a}^2}-s_{\perp a}^2}}{{\pi}^{3/2}v_{th\parallel a} v_{th\perp a}^2},
	\label{eq:bimax}
\ee
%
where $s_{\parallel a}$ and $s_{\perp a}$ are the normalized parallel and perpendicular shifted velocities respectively, defined by
%
\begin{equation}
    s_{\parallel a} = \frac{v_\parallel-u_{\parallel a}}{v_{th\parallel a}},~v_{th\parallel a}^2=\frac{2 T_{\parallel a}}{m_a},
    \label{eq:sparallela}
\end{equation}
%
and
%
\begin{equation}
    s_{\perp a}^2 = \frac{v_\perp^{'2}}{v_{th\perp a}^{2}}=\frac{\mu B}{T_{\perp a}},~v_{th\perp a}^2=\frac{2 T_{\perp a}}{m_a},
    \label{eq:sperpa}
\end{equation}
%
{which provide an efficient representation of the distribution function in both the weak ($u_{\parallel a} \ll v_{th a})$ and strong flow ($u_{\parallel a} \sim v_{th a}$) regimes by better capturing strong near-Maxwellian flows with fewer expansion coefficients \citep{Hirvijoki2016}.}
%


The guiding-center density $N_a$, appearing in \cref{eq:bimax}, the guiding-center fluid velocity $u_{\parallel a}$, in \cref{eq:sparallela}, and the guiding-center parallel $T_{\parallel a}=P_{\parallel a}/N_a$ and perpendicular $T_{\perp a}=P_{\perp a}/N_a$ temperatures in \cref{eq:sparallela,eq:sperpa} are defined as $N_a = ||1||_a$, $N_a u_{\parallel a} = || v_{\parallel}||_a$, $P_{\parallel a} = m_a ||(v_\parallel-u_{\parallel a})^2||_a$, and $P_{\perp a} = ||\mu B ||_a$, where
%
\begin{equation}
    ||\chi||_a \equiv \int \chi \lb F_a \rb_{\mathbf R} \frac{B}{m_a} d\mu dv_\parallel d\theta.
\end{equation}
%
The definition of $N_a$, $u_{\parallel a}$, $P_{\parallel a}$, and $P_{\perp a}$ implies that $N_a^{00}=1,~N_a^{10}=0,~N_a^{20}=0,~N_a^{01}=0,$ respectively.
Later, we will consider the parallel and perpendicular heat fluxes, defined as
%
\begin{align}
        Q_{\parallel a} &= m_a ||(v_\parallel-u_{\parallel a})^3||_a,~Q_{\perp a} = ||(v_\parallel-u_{\parallel a}) \mu B||_a,
        \label{eq:fluidmoments1}
\end{align}
%
which are related to the coefficients $N_a^{30}$ and $N_a^{11}$ by %, N_a^{02}, N_a^{21},$ and $N_a^{40}$ by
%
\be
    \begin{split}
        N_a^{30}&=\frac{Q_{\parallel a}}{\sqrt{3}P_{\parallel a} v_{tha \parallel}},
        ~N_a^{11}=-\frac{\sqrt{2} Q_{\perp a}}{P_{\perp a} v_{tha \parallel}}.
    \end{split}
    \label{eq:kineticmoments1}
\ee


\subsection{Fluid Moment Expansion of the Collision Operator}
\label{section:jifluidexpansion}

A polynomial expansion of the nonlinear Coulomb collision operator in \cref{eq:coulombop} was carried out in \citet{Ji2009}, while the treatment of finite fluid velocity and unlike-species collisions is described in \citet{Ji2008}.
%
This allowed expressing $C_{ab}$ as products of fluid moments of $f_a$ and $f_b$. 
%
We summarize here the main steps of \citet{Ji2006, Ji2008}.
%
For an alternative derivation of the fluid moment expansion in terms of multipole moments of the Coulomb operator, see \cref{ch:op}.

Similarly to \cref{eq:gyrof}, the particle distribution function $f_a$ is expanded as
%
\be
    f_a = f_{aM} \sum_{l,k=0}^{\infty}\frac{L_k^{l+1/2}(c_a^2) \mathbf P^{l}(\mathbf c_a) \cdot {\mathbf M_a}^{lk}}{\sqrt{\sigma_k^l}},
    \label{eq:faji}
\ee

\noindent where  
%
\begin{equation}
    f_{aM}=\frac{n_a}{\pi^{3/2} v_{tha}^3} e^{-c_a^2}
\end{equation}
%
is a shifted Maxwell-Boltzmann distribution function, and $\mathbf c_a$ the shifted velocity defined as $\mathbf c_a=(\mathbf v - \mathbf U_a)/v_{tha}$, with $\mathbf U_a=u_{\parallel a}\mathbf b+\mathbf u_{\perp a}$ the fluid velocity. The fluid variables $n_a, \mathbf U_a,$ and $T_a$ are defined as the usual moments of the particle distribution function $f_a$, i.e. $n_a = \int f_a d^3 \mathbf v$, $n_a \mathbf U_a = \int f_a \mathbf v d^3v$, $n_a T_a = \int m f_a (\mathbf v - \mathbf u_a)^2 d^3v/3$.

The tensors $\mathbf P_a^{lk}(\mathbf c_a)=\mathbf P^{l}(\mathbf c_a)L_k^{l+1/2}(c_a^2)$ constitute an orthogonal basis, where $\mathbf P^l(\mathbf c_a)$ is the symmetric and traceless tensor
%
\be
    \begin{split}
    \mathbf P^l(\mathbf c_a) &= \sum_{i=0}^{\floor{l/2}}d_i^l S_i^l c_a^{2i}\left\{\mathbf I^i \hat c_a^{l-2i}\right\},
    \end{split}
\ee
%
with $\mathbf I$ denoting the identity matrix, $\{\mathbf A^i\}$ denoting the symmetrization of the tensor $\mathbf A^i$, ${\floor{l/2}}$ denoting the largest integer less than or equal to $l/2$, and the coefficients $d_i^l$ and $S_i^l$ defined by
%
\begin{equation}
    d_i^l=\frac{(-2)^i(2l-2i)!l!}{(2l)!(l-i)!},
\end{equation}
%
and
%
\begin{equation}
    S_i^l=\frac{l!}{(l-2i)!2^i i!}.
\end{equation}
%
The tensor $\mathbf P^l(\mathbf c_a)$ is can be also computed using the recursion relation
%
\begin{equation}
    \mathbf P^{l+1}(\mathbf c)=\mathbf c \mathbf P^l(\mathbf c)-\frac{c^2}{2l+1}\frac{\partial \mathbf P^{l}(\mathbf c)}{\partial \mathbf c}
\end{equation}
%
and is normalized via
%
\begin{equation}
    \int d \mathbf v \mathbf P^{n}(\mathbf v)\mathbf P^l(\mathbf v) \cdot \mathbf M^l g(v) = \mathbf M^n \delta_{n,l} \sigma_n \int d \mathbf v v^{2n} g(v),
\label{eq:normplk}
\end{equation}
%
with $\sigma_l = l!/[2^l (l+1/2)!]$.
%
We note that the tensor $\mathbf A^i$ is formed by $i$ multiplications of the $\mathbf A$ elements (e.g., if $\mathbf A$ is a {rank-2 tensor}, $\mathbf A^3 \equiv \mathbf A \mathbf A \mathbf A$, which in index notation can be written as $(\mathbf A^3)_{ijlkmn} = A_{ij}A_{lk}A_{mn}$).

In the expansion in \cref{eq:faji}, $L_k^{l+1/2}(x)$ are the associated Laguerre polynomials
%
\be
    \begin{split}
        L_{k}^{l+1/2}(x)&=\sum_{m=0}^{k}L_{km}^{l} x^m,
    \end{split}
    \label{eq:asslaguerre}
\ee
%
normalized via
%
\begin{equation}
    \int_0^\infty e^{-x} x^{l+1/2} L_k^{l+1/2}(x) L_{k'}^{l+1/2}(x) dx = \lambda_{k}^l\delta_{k,k'}.
\label{eq:normlkl}
\end{equation}
%
with $\lambda_{k}^l={(l+k+1/2)!}/{k!}$ and $L_{km}^{l}=[{(-1)^m(l+k+1/2)!}]/[{(k-m)!(l+m+1/2)!m!}]$. The $\sigma_k^l=\sigma_l \lambda_k^l$ term is a normalization factor from the orthogonality relations in \cref{eq:normplk,eq:normlkl}.
%
Finally, the coefficients of the expansion in \cref{eq:faji} $\mathbf M_a^{lk}$ are
%
\be
    \mathbf M_a^{lk} = \frac{1}{n_a}\int d \mathbf v f_a \frac{L_k^{l+1/2}(c_a^2) \mathbf P^{l}(\mathbf c_a)}{\sqrt{\sigma_k^l}},
    \label{eq:MlkCoulomb}
\ee
%
which correspond to the moments of $f_a$  due to the orthogonality relations in \cref{eq:normplk,eq:normlkl}.

By using the expansion in \cref{eq:faji} in the collision operator in \cref{eq:coulombop}, a closed form for $C_{ab}$ in terms of products of $\mathbf M_a^{lk}$ can be obtained. For like-species collisions it reads
%
\be
    C_{aa}=\sum_{l,k=0}^{\infty}\sum_{n,q=0}^{\infty}\sum_{m=0}^k\sum_{r=0}^q\frac{L_{km}^lL_{qr}^n}{\sqrt{\sigma_k^l \sigma_q^n}}c\left(f_a^{lkm},f_a^{nqr}\right),
    \label{eq:JiCab}
\ee
%
with
%
\be
\begin{split}
    c\left(f_a^{lkm},f_a^{nqr}\right)&=f_{aM}\sum_{u=0}^{\text{min}(2,l,n)}\nu_{*aau}^{lm,nr}(c_a^2)\sum_{i=0}^{\text{min}(l,n)-u}d_i^{l-u,n-u}\mathbf P^{l+n-2(i+u)}(\hat{\mathbf c_a})\cdot ({\mathbf M_a^{lk}\cdot^{i+u}\mathbf M_a^{nq}})_{TS},
\end{split}
    \label{eq:ccjiheld}
\ee
%
where $\hat{\mathbf c_a} = \mathbf c_a/c_a$, $\cdot^n$ is the $n$-fold inner product (e.g., for the matrix $\mathbf A = A_{ij}$, $(\mathbf A \cdot^1 \mathbf A)_{ij} = \sum_k A_{ki}A_{kj}$), and $({\mathbf A})_{TS}$ the traceless symmetrization of $\mathbf A$ (e.g., $({\mathbf A})_{TS} = (A_{ij}+A_{ji})/2-\delta_{ij}\sum_k A_{kk}/3$).
%
We refer the reader to \citet{Ji2009} for the explicit form of the $\nu_{*abu}^{lm,nr}$ coefficients.

\subsection{Guiding-Center Moment Expansion of the Collision Operator}
\label{section:cabmomentexpansion}

In order to apply the gyroaveraging operator to the like-species collision operator $C_{aa}$ in \cref{eq:JiCab}, we expand the fluid moments as $\mathbf M_{a}^{lk}=\mathbf M_{a0}^{lk} + \epsilon \mathbf M_{a1}^{lk} + ...$, aiming at representing the collision operator up to $O(\epsilon_\nu \epsilon)$.
An analytical expression for the leading-order $\mathbf M_{a0}^{lk}$ in terms of guiding-center moments $N_{a}^{pj}$ can be obtained as follows.
By splitting $f_a = \lb f_a \rb_{\mathbf R} + \tilde f_a$ when evaluating the fluid moments $\mathbf M_a^{lk}$ according to \cref{eq:MlkCoulomb}, we obtain
%
\begin{equation}
    \mathbf M_a^{lk} = \frac{1}{n_a}\int d^3 x' d^3 v' \delta(\mathbf x' - \mathbf x)\frac{L_k^{l+1/2}(c_a^{'2}) \mathbf P^{l}(\mathbf c'_a)}{\sqrt{\sigma_k^l}}\left(\lb f_a \rb_{\mathbf R} + \tilde f_a\right).
    \label{eq:malkexact1}
\end{equation}
%
where the Dirac delta function was introduced to convert the velocity integral into an $(\mathbf x, \mathbf v)$ integral that encompasses the full phase-space.
Since the volume element in phase space can be written as $d^3 \mathbf x d^3\mathbf v = (B_\parallel^*/m)d \mathbf R dv_\parallel d \mu d\theta$ \citep{Cary2009}, and defining $\mathbf x' = \mathbf R + \rho_a \mathbf a$, we can write the fluid moments in \cref{eq:malkexact1} as
%
\begin{equation}
\begin{split}
    \mathbf M_a^{lk} &= \frac{1}{n_a}\int d \mathbf R  d v_\parallel d\mu d\theta\frac{B_\parallel^*}{m_a} \delta(\mathbf x-\mathbf R -  \rho_a \mathbf a)\frac{L_k^{l+1/2}(c_a^{'2}) \mathbf P^{l}(\mathbf c'_a)}{\sqrt{\sigma_k^l}}\left( \lb F_a \rb_{\mathbf R} + \tilde F_a\right).    
\end{split}
\label{eq:malkexact}
\end{equation}
%
where $\lb f_a \rb_{\mathbf R}$ and $\tilde f_a$ in \cref{eq:malkexact1} are written in terms of guiding-center coordinates using \cref{eq:fguidF}.
Neglecting the higher-order $\mathbf \rho_a$ and $\tilde F_a$ terms, the leading-order fluid moments $\mathbf M_{a0}^{lk}$ are given by
%
\begin{equation}
    \mathbf M_{a0}^{lk} = \frac{1}{n_a}\int d v_\parallel d\mu d\theta\frac{B_\parallel^*}{m_a} \frac{L_k^{l+1/2}(c_a^{'2}) \mathbf P^{l}(\mathbf c'_a)}{\sqrt{\sigma_k^l}}\lb F_a \rb_{\mathbf R} .
    \label{eq:malk0exact}
\end{equation}
%
The $\theta$ integration can be performed by making use of the gyroaveraging formula of the $\mathbf P^l$ tensor
%
\be
    \lb \mathbf P^l(\mathbf c_a) \rb_{\mathbf R} = c_a^{l} P_l\left(\xi_a\right) \mathbf P^l(\mathbf b),
    \label{eq:Pgyro}
\ee
%
where $\xi_a=\mathbf c_a \cdot \mathbf b/c_a$ is the pitch angle velocity coordinate, and $P_l$ is a Legendre polynomial defined by
%
\begin{equation}
    P_l(x)=\frac{1}{2^ll!}\frac{d^l}{dx^l}\left[(x^2-1)^l\right],
\end{equation}
%
and normalized via
%
\begin{equation}
    \int_{-1}^1 P_l(x)P_{l'}(x)dx=\frac{\delta_{ll'}}{l+1/2},
\end{equation}
%
yielding
%
\begin{equation}
    \mathbf M_{a0}^{lk} = \frac{\mathbf P^l(\mathbf b)}{n_a}\int d v_\parallel d\mu  d\theta \frac{B_\parallel^*}{m_a} \frac{L_k^{l+1/2}(c_a^{'2}) c_a^{l} P_l\left(\xi_a\right) }{\sqrt{\sigma_k^l}}\lb F_a \rb_{\mathbf R}.
    \label{eq:malk0exact1}
\end{equation}
%
For the derivation of \cref{eq:Pgyro}, see \cref{sec:gudingcentertransf}.
%
Finally, we use the basis transformation
%
\be
    \begin{split}
        c_a^l P_l(\xi_a)L_k^{l+1/2}(c_a^2)=&\sum_{p=0}^{l+2k}\sum_{j=0}^{k+\floor{l/2}}T_{alk}^{pj}
        H_p(s_{\parallel a})L_j(s_{\perp a}^2),
    \end{split}
    \label{eq:tlkpj}
\ee
%
with the inverse
%
\be
    \begin{split}
        H_p(s_{\parallel a})L_j(s_{\perp a}^2) =& \sum_{l=0}^{p+2j}\sum_{k=0}^{j+\floor{p/2}}\left(T_a^{-1}\right)_{pj}^{lk}
        c_a^l P_l(\xi_a)L_k^{l+1/2}(c_a^2),
    \end{split}
    \label{eq:tminus1pjlk}
\ee
%
to obtain an expression for the integrand in \cref{eq:malk0exact1} in terms of the Hermite-Laguerre basis.
A numerical evaluation of  $T_{alk}^{pj}$ and $\left(T_a^{-1}\right)_{pj}^{lk}$ was carried out in \citet{Omotani2015}.
%
Instead, in Appendix \ref{app:tlkpj}, we derive {the} analytic expressions {of both $T_{alk}^{pj}$ and $\left(T_a^{-1}\right)_{pj}^{lk}$}.
%
Using the definition of guiding-center moments $N_a^{pj}$ in \cref{eq:gyromoments}, the leading-order fluid moment $\mathbf M_{a0}^{lk}$ is then given by
%
\be
    n_a \mathbf M_{a0}^{lk} = N_a \mathbf P^l(\mathbf b) \mathcal{N}_a^{lk},
    \label{eq:CoulDKmom}
\ee
%
where we define
%
\begin{equation}
    \mathcal{N}_a^{lk} = \sum_{p=0}^{l+2k}\sum_{j=0}^{k+\floor{l/2}}T_{alk}^{pj}{N}_a^{pj}\sqrt{\frac{2^p p!}{\sigma_k^l}}.
\label{eq:CoulDKmom1}
\end{equation}

The leading-order part $C_{aa0}$ of the collision operator $C_{aa}$ can be calculated by approximating $\mathbf M_{a}^{lk}$ appearing in \cref{eq:ccjiheld} with $\mathbf M_{a0}^{lk}$.
For the ions, the largest contribution to $\mathbf M_{i}^{lk}-\mathbf M_{i0}^{lk}$ is of order $\epsilon$ and it is given by the $\rho_i$ appearing in \cref{eq:malkexact} [the $\tilde F_i$ correction is smaller since $\tilde F_i < \epsilon^2 \lb F_i \rb_{\mathbf R}$, see \cref{eq:orderingftildei}]. Therefore, by using the ordering in \cref{eq:orderingnu2}, the largest correction to $C_{ii0}$ is $O(\sqrt{m_e/m_i} \epsilon \epsilon_\nu)$. The correction to $C_{ee0}$ is of the same order. It follows that we can approximate $C_{aa}$ appearing in \cref{eq:ccjiheld} with $C_{aa0}$ to represent the collision operator up to $O(\epsilon_{\nu} \epsilon)$.

As an aside, we note that the relationship between the guiding-center and fluid moments in \cref{eq:CoulDKmom} provides, for the indices $(l,k)=(0,0)$,
%
\begin{align}
    n_a &= N_a,
\end{align}
%
while, for $(l,k)=(0,1)$, yields
%
\begin{align}
    T_a = \frac{T_{\parallel a}+2 T_{\perp a}}{3}.
\end{align}
%
Moreover, the $(l,k)=(2,0)$ moment provides a relationship useful to express the viscosity tensor $\mathbf \Pi_a = \int (\mathbf c_a \mathbf c_a - c_a^2 \mathbf I) f_a d\mathbf v$ as
%
\begin{align}
    \mathbf \Pi_a = \mathbf b \mathbf b N (T_{\parallel a} - T_{\perp a}),
\label{eq:stressdk}
\end{align}
%
while for $(l,k)=(1,1)$ gives
%
\begin{align}
    \mathbf q_a &= \left(\frac{Q_{\parallel a}}{2}+Q_{\perp a}\right)\mathbf b,
\label{eq:heatfluxdk}
\end{align}
%
with $\mathbf q_a$ the heat flux density $\mathbf q_a = m \int \mathbf c_a c_a^2 f_a d \mathbf v/2$.

In order to express the Boltzmann equation, \cref{eq:boltzmannGC}, in terms of the guiding-center moments $N_a^{pj}$, we evaluate the guiding-center moments of $\lb C_{aa} \rb_{\mathbf R}$ which, up to $O(\epsilon^2)$, are given by
\be
    \begin{split}
         C_{aa}^{pj}=\frac{1}{N_a} \int \lb C_{aa0} \rb_{\mathbf R}  \frac{H_p(s_{\parallel a}) L_j(s_{\perp a}^2)}{\sqrt{2^p p!}} \frac{B}{m_a} dv_{\parallel} d\mu d\theta.
        \label{eq:CoulIntGyro}
    \end{split}
\ee
%
By using the gyroaveraging property of $\mathbf P^{l}(\mathbf c_a)$ in \cref{eq:Pgyro} in the like-species operator in \cref{eq:JiCab,eq:ccjiheld} (with $\mathbf M_{a}^{lk} = \mathbf M_{a0}^{lk}$), and the relation between $\mathbf M_{a0}^{lk}$ and $N_{a}^{pj}$ in \cref{eq:CoulDKmom}, the gyroaveraged collision operator coefficients $\lb c\left(f_a^{lkm},f_a^{nqr}\right) \rb_{\mathbf R}$ are given by
%
\be
    \begin{split}
    &\lb c(f_a^{lkm},f_a^{nqr})\rb_{\mathbf R}=f_{aM}\sum_{u=0}^{\text{min}(2,l,n)}\nu_{*aau}^{lm,nr}(c_a^2)\sum_{i=0}^{\text{min}(l,n)-u}d_i^{l-u,n-u} P_{l+n-2(i+u)}(\xi) \mathcal{N}_a^{lk}\mathcal{N}_a^{nq}\mathcal{P}^{l,n}_{i+u},
    \end{split}
    \label{eq:JiCabGyro}
\ee
%
with $\mathcal{P}^{l,n}_{i+u}=\mathbf{P}^{l+n-2(i+u)}\cdot ({\mathbf P^{l}\cdot^{i+u}{\mathbf P}^{n}})_{TS}$.
%
Using the basis transformation of \cref{eq:tminus1pjlk} to express ${H_p(s_{\parallel a}) L_j(s_{\perp a}^2)}$ in \cref{eq:CoulIntGyro} in terms of $c_a^l P_l(\xi_a)L_k^{l+1/2}(c_a^2)$, and performing the resulting integral, we obtain
%
\be
    \begin{split}
        C_{aa}^{pj}=&\sum_{l,k}\sum_{n,q}\sum_{u=0}^{\text{min}(2,l,n)}\sum_{i=0}^{\text{min}(l,n)-u}\sum_{e=0}^{p+2j}\sum_{f=0}^{j+\floor{p/2}}\sum_{g=0}^f\sum_{m=0}^k\sum_{r=0}^q\\
        &\frac{L_{km}^lL_{qr}^n L_{fg}^e d_i^{l-u,n-u}}{\sqrt{\sigma_k^l \sigma_q^n} (e+1/2)4 \pi}\frac{C_{*aau}^{eg,lm,nr}}{\sqrt{2^p p!}}\delta_{e,l+n-2(i+u)}{\left(T^{-1}\right)}_{pj}^{ef} \mathcal{N}_a^{lk}\mathcal{N}_a^{nq}\mathcal{P}^{l,n}_{i+u},
    \end{split}
    \label{eq:caapjexact}
\ee
%
with $C_{*aabu}^{jw,lm,nr}=\int d\mathbf v c_a^{2w+j}f_{Ma}\nu_{*aau}^{lm,nr}$ [for an efficient algorithmic representation of $C_{*aabu}^{jw,lm,nr}$ see \citet{Ji2009}].

We now turn to the electron-ion collision operator, $C_{ei} = C_{ei}^0 + C_{ei}^1$, with $C_{ei}^0$ given by \cref{eq:cei0} and $C_{ei}^1$ given by \cref{eq:cei1}.
As the basis $ L_k^{l+1/2} \mathbf P^l(\mathbf c_a)$ is an eigenfunction of the Lorentz pitch-angle scattering operator $C_{ei}^0$ with eigenvalue $-l(l+1)$ \citep{Ji2008}, we write $C_{ei}^0$ as
%
\be
    C_{ei}^0=-\sum_{l,k}\frac{n_i L_{ei}}{v_{the}^3 c_e^3}\frac{l(l+1)f_{eM}}{\sqrt{\sigma_k^l}}L_k^{l+1/2}(c_e^2) \mathbf P^l(\mathbf c_e) \cdot {\mathbf M_e}^{lk}.
    \label{eq:cei0eig}
\ee
%
Similarly to like-species collisions, we approximate $\mathbf M_{e}^{lk} \simeq \mathbf M_{e0}^{lk}$ in \cref{eq:cei0eig}, representing $C_{ei}^0$ accurately up to $O(\epsilon_\nu \epsilon)$.
Using the basis transformation in \cref{eq:tminus1pjlk} and the gyroaverage property of $\mathbf P^l (\mathbf c_a)$ in \cref{eq:Pgyro}, we take guiding-center moments of $C_{ei}$ of the form (\ref{eq:CoulIntGyro}), and obtain
%
\be
\begin{split}
    C_{ei}^{pj} = -\frac{\nu_{ei}}{8 \pi^{3/2}}&
    \sum_{l=0}^{p+2j}\sum_{f=0}^{j+\floor{p/2}}
    \frac{{\left(T^{-1}_e\right)}_{pj}^{lf}}{\sqrt{2^p p!}}\left[\sum_{k=0}^\infty A_{ei}^{lf,k} \mathcal{N}_e^{lk} -\delta_{l,1}\frac{{u_{\parallel i}}}{v_{the}}\frac{{16}}{3 } \frac{\Gamma(f+3/2)}{f!\sqrt{\pi}}\right],
\end{split}
\label{eq:ceipj}
\ee
%
where the $A_{ei}$ coefficients are given by
%
\be
    \begin{split}
        A_{ei,0}^{lf,k}=&\frac{l(l+1)}{l+1/2}\frac{(l!)^2 2^l}{(2l)!} \sum_{m=0}^f \sum_{n=0}^k \frac{L_{fm}^l L_{kn}^l}{\sqrt{\sigma_k^l}} {(l+m+n-1)!},
    \end{split}
\ee
%
where we used the identity $|\mathbf P^l(\mathbf b)|^2=2^l (l!)^2/(2l)!$ \citep{Snider2018}.

Finally, for the ion-electron collision operator, $C_{ie}$, we neglect $O(\sqrt{m_e/m_i}\epsilon_\nu \epsilon)$ corrections by approximating $F_i \simeq \lb F_i \rb_{\mathbf R}$, and use the transformation in \cref{eq:GCcoordinates} to convert the $C_{ie}$ operator in \cref{eq:cie} to guiding-center variables, yielding
%
\begin{align}
    C_{ie} &= \frac{\mathbf R_{ei}}{m_i n_i v_{th i }}\cdot\left[\mathbf c_\perp \frac{m_i v_{thi}^2}{B}\frac{\partial \lb F_i \rb_{\mathbf R}}{\partial \mu}+\mathbf b \frac{\partial \lb F_i \rb_{\mathbf R}}{\partial c_{\parallel i}}\right]+\nu_{ei}\frac{m_e}{m_i}\frac{n_e}{n_i}\bigg[3 \lb F_i \rb_{\mathbf R}\nonumber\\
    &\l. + c_{\parallel i} \frac{\partial \lb F_i \rb_{\mathbf R}}{\partial c_{\parallel i}} +2\mu \frac{\partial \lb F_i \rb_{\mathbf R}}{\partial \mu}+ \frac{T_e}{2T_i}\frac{\partial^2 \lb F_i \rb_{\mathbf R}}{\partial c_{\parallel i }^2}+\frac{2 T_e}{B} \frac{\partial}{\partial \mu}\left(\mu \frac{\partial \lb F_i \rb_{\mathbf R}}{\partial \mu}\right)\right].
    \label{eq:cie11}
\end{align}
%
By evaluating $\mathbf R_{ei}$ at the guiding-center position $\mathbf R$ (neglecting higher order $\epsilon$ effects), we write $\mathbf R_{ei} \cdot \mathbf b = N_e m_e v_{th\parallel e} C_{ei}^{10}/\sqrt{2} + O(\sqrt{m_e/m_i}\epsilon_\nu \epsilon)$ and gyroaverage \cref{eq:cie11}, yielding
%
\begin{equation}
    \begin{split}
        \lb C_{ie} \rb_{\mathbf R}&=\frac{C_{ei}^{10}}{\sqrt{2}}\frac{m_e}{m_i} \frac{N_e}{n_i} \frac{v_{th\parallel e}}{v_{th\parallel i}} \frac{\partial \lb F_i\rb_{\mathbf R} }{\partial s_{\parallel}}+\nu_{ei}\frac{m_e}{m_i}\frac{n_e}{n_i}\bigg[3 \lb F_i \rb_{\mathbf R}\\
        %\lb C_{ie}^1 \rb_{\mathbf R}&=
        &\l.+ s_{\parallel i} \frac{\partial \lb F_i \rb_{\mathbf R}}{\partial s_{\parallel i}}+2\mu \frac{\partial \lb F_i \rb_{\mathbf R}}{\partial \mu}+ \frac{T_e}{2 T_{\parallel i}}\frac{\partial^2 \lb F_i \rb_{\mathbf R}}{\partial s_{\parallel i}^2} +\frac{2 T_e }{B}\frac{\partial}{\partial \mu}\left(\mu \frac{\partial \lb F_i \rb_{\mathbf R}}{\partial \mu}\right) \right],
    \end{split}
    \label{eq:gyrocie}
\end{equation}
%
where we used $c_{\parallel i}^2 = s_{\parallel i}^2 T_{\parallel i}/T_{i}$.
Taking guiding-center moments of the form (\ref{eq:CoulIntGyro}) of $\lb C_{ie} \rb_{\mathbf R}$ in \cref{eq:gyrocie}, we obtain
%
\begin{equation}
    C_{ie}^{pj}=\nu_{ei}\frac{m_e}{m_i}\sum_{lk}B_{lk}^{pj}N_{i}^{lk},
    \label{eq:ciepj}
\end{equation}
%
with
%
\begin{equation}
\begin{split}
    B_{lk}^{pj}&=2j\delta_{lp}\delta_{kj-1}\left(1-\frac{T_e}{T_{\perp i}}\right)-\sqrt{ p}\frac{v_{th\parallel e}}{v_{th\parallel i}}\frac{C_{ei}^{10}}{\nu_{ei}}\delta_{lp-1}\delta_{kj}\\
    &-(p+2j)\delta_{lp}\delta_{kj}+\sqrt{ p (p-1)}\delta_{l p-2}\delta_{kj}\left(\frac{T_e}{T_{\parallel i}}-1\right).
\end{split}
\end{equation}

\section{Drift-Kinetic Moment-Hierarchy}
\label{sec:momenthierarchy}

In this section, we derive a set of equations that describe the evolution of  the guiding-center moments $N_a^{pj}$, by integrating in guiding-center velocity space the conservative form of the Boltzmann equation, \cref{eq:boltzmannGC}, with the weights $H_p(s_{\parallel a})L_j(s_{\perp a}^2)$.
First, we highlight the dependence of $\dot{\mathbf R}$ and $\dot v_{\parallel}$ on $s_{\parallel a}$ and $s_{\perp a}^2$ by rewriting the equations of motion as
%
\begin{equation}
\begin{split}
    \dot{\mathbf R} &= \mathbf U_{0 a} + \mathbf U_{p a}^* + s_{\perp a}^2 \mathbf U_{\nabla B a}^* + s_{\parallel a}^2 \mathbf U_{k a}^* + s_{\parallel a}(v_{th\parallel a} \mathbf b + \mathbf U_{p a}^{*th}),
\end{split}
\label{eq:rdotGCform}
\end{equation}
%
and
%
\begin{equation}
\begin{split}
    m_a \dot{v}_\parallel &= F_{\parallel a}-s_{\perp a}^2 F_{M a} +s_{\parallel a} F_{p a}^{th}-m_a \mathcal{A}.
\end{split}
\label{eq:vparGCform}
\end{equation}
%
In \cref{eq:rdotGCform,eq:vparGCform}, $\mathbf U_{0 a} = \mathbf v_E + u_{\parallel a} \mathbf b$ is the lowest-order guiding-center fluid velocity, $\mathbf U_{\nabla B a}^* = (T_{\perp a}/m_a)(\mathbf b \times \nabla B/\Omega_a^{*} B)$ is the fluid grad-B drift, with $\Omega_a^{*} = q_a B_{\parallel}^* / m_a$, $\mathbf U_{ka}^{*} = (2 T_{\parallel a}/m_a)(\mathbf b \times \mathbf k/\Omega_a^*)$ is the fluid curvature drift with $\mathbf k = \mathbf b \cdot \nabla \mathbf b$, $\mathbf U_{pa}^* = ({\mathbf b}/{\Omega_a^*})\times d_0 \mathbf U_{0 a}/dt$ is the fluid polarization drift, $F_{\parallel a} = q_a E_\parallel+m_a\mathbf v_E \cdot d_0 \mathbf b/{dt}$, $F_{M a} = {T_{\perp a}}{}\nabla_\parallel \ln B$ is the fluid mirror force, and both $\mathbf U_{p a}^{*th}$ and $F_{p a}^{th}$ are related to gradients of the electromagnetic fields
%
\begin{equation}
\begin{split}
    \mathbf U_{p a}^{*th} &= v_{th\parallel a}\frac{\mathbf b}{\Omega_a^*}\times \left(\mathbf b \cdot \nabla \mathbf v_E+\mathbf v_E \cdot \nabla \mathbf b + 2 u_{\parallel a} \mathbf k\right),\\
    F_{p a}^{th} &=  m_a v_{th\parallel a} \mathbf E \cdot \left(\frac{\mathbf b \times \mathbf k}{B}\right).
\end{split}
\end{equation}
%
The fluid convective derivative operator is defined as 
%
\begin{equation}
    \frac{d_{0 a}}{dt} = \partial_t + \mathbf U_{0 a} \cdot \nabla.  
    \label{eq:convdev0}
\end{equation}

Next, to obtain an equation for the moment $N_a^{pj}$, we apply the guiding-center moment operator
%
\begin{equation}
\begin{split}
    ||\chi||_a^{*pj} &=\frac{1}{N_a B} ||\chi H_p(s_{\parallel a}) L_j(s_{\perp a}^2) B_{\parallel}^*||\\
    &= \frac{1}{N_a}\int \chi \frac{B_\parallel^*}{m_a} \lb F_a \rb_{\mathbf R} \frac{H_p(s_{\parallel a}) L_j(s_{\perp a}^2)}{\sqrt{2^p p!}}  dv_\parallel d\mu d\theta,
\end{split}
\end{equation}
%
to Boltzmann's equation, \cref{eq:boltzmannGC}. By defining $|| 1 ||_a^{*pj} = {N}_a^{*pj}$ such that
%
\begin{equation}
\begin{split}
    {N}_a^{*pj} &= N_a^{pj}\left(1+\frac{\mathbf b \cdot \nabla \times \mathbf v_E}{\Omega_a}+u_{\parallel a} \frac{\mathbf b \cdot \nabla \times \mathbf b}{\Omega_a}\right)\\
    &+ v_{th\parallel a}\frac{\mathbf b \cdot \nabla \times \mathbf b}{\sqrt{2}\Omega_a}\left(\sqrt{p+1}N_a^{p+1~j}+\sqrt{p}N_a^{p-1~j}\right),
\end{split}
\label{eq:overlinenapj}
\end{equation}
%
and
%
\begin{equation}
\begin{split}
    \frac{d_a^{*pj}}{dt}={N}_a^{*pj}\frac{\partial}{\partial t}+\left|\left|\dot{\mathbf R}\right|\right|_a^{*pj} \cdot \nabla,
\end{split}
\end{equation}
%
the drift-kinetic moment-hierarchy conservation equation for species $a$ is
%
\be
    \begin{split}
        \frac{\partial {N}_a^{*pj}}{\partial t} + \nabla \cdot \left|\left|{\dot{\mathbf R}}\right|\right|_a^{*pj}-\frac{\sqrt{2 p}}{v_{th\parallel a}} \left|\left|\dot v_\parallel\right|\right|_a^{*p-1j} +\mathcal{F}_a^{pj}= \sum_b C_{ab}^{pj},
    \end{split}
    \label{eq:finalDKE}
\ee
%
where we define the fluid operator
%
\be
    \begin{split}
        \mathcal{F}_a^{pj} &= \frac{d_a^{*pj}}{dt}\ln\left(N_a T_{\parallel a}^{p/2} T_{\perp a}^jB^{-j}\right)+\frac{\sqrt{2p}}{v_{th\parallel a}}\frac{d^{*p-1 j}u_{\parallel a}}{dt}\\
        &+\frac{\sqrt{p(p-1)}}{2}\frac{d_a^{*p-2 j}}{dt}\ln T_{\parallel a}-j\frac{d_a^{*pj-1}}{dt}\ln\left(\frac{ T_{\perp a}}{B}\right),
    \end{split}
    \label{eq:finalDKEF}
\ee
%
since it is the key term that describes the evolution of the guiding-center fluid properties $N_a, u_{\parallel a}, P_{\perp a},$ and $P_{\parallel a}$ (see \cref{sec:fluidmodel}).
%
The guiding-center moments of the particle's equations of motion are given by
%
\be
    \begin{split}
        \left|\left|{\dot{\mathbf R}}\right|\right|_a^{*pj}&= \sum_{l,k}\left(\mathbf U_{0 a}\delta_{pl}\delta_{jk}  + v_{th\parallel a}\mathbf b\mathcal{V}_{lk}^{1pj}\right){N}_a^{*lk}\\
        & +\left(\mathbf U_{pa}\delta_{pl}\delta_{jk} + \mathbf U_{pa}^{th} \mathcal{V}_{lk}^{1pj}+ \mathbf U_{\nabla B a}\mathcal{M}_{lk}^{pj} + \mathbf U_{k a}\mathcal{V}_{lk}^{2pj}\right) N_a^{lk},
    \end{split}
    \label{eq:finalDKE3}
\ee
\be
    \begin{split}
        m_a\left|\left|{\dot v_\parallel}{}\right|\right|_a^{*pj}&=\sum_{l,k}\left[F_{\parallel a} \delta_{p,l}\delta_{j,k} +  F_{p a}^{th} \mathcal{V}_{lk}^{1pj} +F_{M a}\mathcal{M}_{lk}^{pj}\right] {N}_a^{*lk}+m_a\left|\left| \mathcal{A}\right|\right|_a^{*pj}.
    \end{split}
    \label{eq:finalDKE4}
\ee
%
where the phase-mixing operators read
%
\begin{align}
    \mathcal{V}_{lk}^{1pj}&=\left(\sqrt{\frac{p+1}{2}}\delta_{p+1, l}+\sqrt{\frac{p}{2}}\delta_{p-1 ,l}\right)\delta_{k,j},\label{eq:finalDKE5}\\
    \mathcal{V}_{lk}^{2pj}&=\left[\delta_{p,l}\left(p+\frac{1}{2}\right)+\frac{\sqrt{(p+2)(p+1)}}{2}{\delta_{p+2 ,l}}{}+ \frac{\sqrt{p(p-1)}}{2}\delta_{p-2,l} \right]\delta_{j,k},\label{eq:finalDKE6}\\
    \mathcal{M}_{lk}^{pj}&=(2j+1)\delta_{p,l}\delta_{j,k}-(j+1)\delta_{p,l}\delta_{j+1, k}-j\delta_{p,l}\delta_{j-1 ,k}.\label{eq:finalDKE7}
\end{align}
%
The expressions of $\mathbf U_{p a}, U_{\nabla B_a}, U_{p a}^{th}$, and $\mathbf U_{k a}$ are derived from $\mathbf U_{p a}^{*}, U_{\nabla B a}^{*}, U_{p a}^{*th}$, and $\mathbf U_{k a}^{*}$ by replacing $\Omega_a^*$ with $\Omega_a$.

The expression of $\left|\left| \mathcal{A}\right|\right|^{*pj}$ 
In \cref{eq:finalDKE4} is given by
%
\begin{equation}
\begin{split}
    || \mathcal{A}||_a^{*pj} &= \frac{1}{N_a \Omega_a} \sum_{l,k}\left(A_{1 a}\mathcal{V}_{lk}^{3pj}+A_{2 a} \mathcal{V}_{lk}^{2pj}+A_{3 a} \mathcal{V}_{lk}^{1pj}\right.\\
    &\left.+A_{4 a} \mathcal{V}_{lk}^{1p'j'}\mathcal{M}_{p'j'}^{pj}+A_{5 a} \mathcal{M}_{lk}^{pj}+A_{6 a} \delta_{pl}\delta_{jk}\right)N_a^{lk},
\end{split}
\label{eq:mathavv}
\end{equation}
%
with the phase-mixing term
%
\be
    \begin{split}
        \mathcal{V}_{lk}^{3pj}=&\left[\sqrt{(p+3)(p+2)(p+1)}\delta_{p+3,l}+3\sqrt{(p+1)^3}{\delta_{p+1 ,l}}\right.\\
        &\left.+ 3\sqrt{p^3}{\delta_{p-1 ,l}}+\sqrt{p(p-1)(p-2)}\delta_{p-3,l} \right]\frac{\delta_{j,k}}{\sqrt{8}},
    \end{split}
    \label{eq:vv3pjlk}
\ee
%
and the coefficients $A_{ia}$
%
\begin{align}
        A_{1 a}&= v_{th\parallel a}^3 \nabla_\perp \cdot \nabla \times \mathbf b,\\
        A_{2 a} &= v_{th\parallel a}^2 \left[\nabla_\perp \cdot (u_{\parallel a} \nabla \times \mathbf b + \nabla \times \mathbf v_E) +  \nabla \times \mathbf b \cdot \mathbf A_a\right],\\
        A_{3 a} &=v_{th\parallel a} (u_{\parallel a} \nabla \times \mathbf b + \nabla \times \mathbf v_E) \cdot \mathbf A_a + v_{th\parallel a}^2 \nabla \times \mathbf b \cdot \mathbf C,\\
        A_{4 a} &= v_{th\parallel a} \frac{T_\perp}{m_a B}\nabla_\perp B \cdot \nabla \times \mathbf b,\\
        A_{5 a} &= \frac{T_{\perp a}}{m_a B}\nabla_\perp B \cdot (u_{\parallel a} \nabla \times \mathbf b + \nabla \times \mathbf v_E),\\
        A_{6 a} &= (v_{th\parallel a} u_{\parallel a} \nabla \times \mathbf b + \nabla \times \mathbf v_E) \cdot \mathbf C
\end{align}
%
with
%
\begin{align}
        \mathbf A_a &= \bm b \times \left[ \frac{\partial \mathbf b}{\partial t} + (\mathbf b \cdot \nabla) \mathbf v_E + (\mathbf v_E \cdot \nabla) \mathbf b +2 u_{\parallel a} v_{th\parallel a} \mathbf k\right]\times \bm b,\\
        \mathbf C &= \frac{\bm b \times}{v_{th\parallel a}}\left[\frac{\partial \mathbf v_E}{\partial t}+(\mathbf v_E \cdot \nabla)\mathbf v_E + u_{\parallel a}^2 \mathbf k\right]\times \bm b.
\end{align}

{Similar moment-hierarchy models (with uniform magnetic fields) have been numerically implemented, and successfully compared with their kinetic counterpart \citep{Paskauskas2009,Loureiro2015,Schekochihin2016,Groselj2017}, and even shown to be more efficient than other velocity discretization techniques in the same region of validity \citep{Camporeale2016}. Equation (\ref{eq:finalDKE}) generalizes such models to spatially varying fields and full Coulomb collisions, while retaining phase-mixing operators that couple nearby Hermite and Laguerre moments and providing a close form for the projection of the Coulomb operator in velocity space. We also note that the use of shifted velocity polynomials in the Hermite-Laguerre basis, which gives rise to the fluid operator $\mathcal{F}_a^{pj}$, allows us to have an efficient representation of the distribution function both in the weak ($u_{\parallel a} \ll v_{th a})$ and strong flow ($u_{\parallel a} \sim v_{th a}$) regimes. As we will see in \cref{sec:fluidmodel}, the fluid operator $\mathcal{F}_a^{pj}$ generates the lowest order fluid equations, as it is present even if all kinetic moments $N_a^{pj}$ (except $N_a^{00}$) are set to zero.}

\section{Drift-Kinetic Poisson's Equation}
\label{sec:poisson}

We use Poisson's equation to evaluate the electric field appearing in the moment-hierarchy equation, \cref{eq:finalDKE}.
In $(\mathbf x, \mathbf v)$ coordinates, Poisson's equation reads
%
\begin{equation}
\begin{split}
        \epsilon_0 \nabla \cdot \mathbf E &= \sum_a q_a n_a=\sum_a q_a \int f_a d^3 v.
\end{split}
\label{eq:fmoment}
\end{equation}
%
Following the same steps used to derive \cref{eq:malkexact} from \cref{eq:MlkCoulomb}, we can write Poisson's equation, \cref{eq:fmoment}, as
%
\begin{equation}
    \epsilon_0 \nabla \cdot \mathbf E = \sum_a q_a \int d^3 \mathbf R dv_\parallel d\mu d\theta \frac{B_\parallel^*}{m} \delta(\mathbf R + \rho_a \mathbf a - \mathbf x)F_a(\mathbf R, v_\parallel, \mu, \theta).
    \label{eq:poissonexact1}
\end{equation}
%
Equation (\ref{eq:poissonexact1}) shows that all particles that have a Larmor orbit crossing a given point $\mathbf x$, give a contribution to the charge density at this location.

Performing the integral over $\mathbf R$ and introducing the Fourier transform $F_a(\mathbf x - \rho_a \mathbf a,v_\parallel, \mu, \theta) = \int d^3\mathbf k F_a(\mathbf k,v_\parallel, \mu, \theta) e^{-i \mathbf k \cdot \mathbf x} e^{i \rho_a \mathbf k \cdot \mathbf a}$, \cref{eq:poissonexact1} can be rewritten as
%
\begin{equation}
    \epsilon_0 \nabla \cdot \mathbf E = \sum_a q_a \int  dv_\parallel d\mu d^3 \mathbf k d\theta \frac{B_\parallel^*}{m_a} F_a(\mathbf k, v_\parallel, \mu, \theta)e^{-i \mathbf k \cdot \mathbf x} e^{i \rho_a \mathbf k \cdot \mathbf a}.
    \label{eq:poissonexact2}
\end{equation}
%
To perform the $\mathbf k$ integration, we use the cylindrical coordinate system $(k_{\perp}, \alpha, k_\parallel)$, expressing $\mathbf k = k_\perp \cos \theta \mathbf e_1+k_\parallel \mathbf b$, such that $\mathbf k \cdot \mathbf a = k_\perp \cos \theta$. This coordinate system allows us to express $e^{i  \rho_a \mathbf k \cdot \mathbf a}$ in \cref{eq:poissonexact2} in terms of Bessel functions using the Jacobi-Anger expansion \citep{Andrews1992}
%
\begin{equation}
    e^{i k_\perp \rho_a \cos \theta} = J_0(k_\perp \rho_a) + 2 \sum_{l=1}^{\infty} J_l(k_\perp \rho_a) i^l \cos (l \theta) = \sum_{l=-\infty}^{\infty}i^l J_l(k_\perp \rho_a)e^{i l \theta},
\label{eq:jacobianger}
\end{equation}
%
where $J_l(k_\perp \rho_a)$ is the Bessel function of the first kind of order $l$. We can then write
%
\begin{equation}
\begin{split}
    \epsilon_0 \nabla \cdot \mathbf E = \sum_a &q_a \int  dv_\parallel d\mu d\theta \frac{B_\parallel^*}{m_a}  \left(  \Gamma_0[F_a]+  2\sum_{l=1}^{\infty} i^l \Gamma_l[F_a \cos(l\theta) ]\right).    
\end{split}
\label{eq:poissonotexact3}
\end{equation}
%
where the Fourier-Bessel operator $\Gamma_l[f]$ is defined as
%
\begin{equation}
    \Gamma_l[F_a(\mathbf k, v_\parallel, \mu, \theta)] \equiv \int d^3 \mathbf k J_l(k_\perp \rho_a) F_a(\mathbf k, v_\parallel, \mu, \theta) e^{-i \mathbf k \cdot \mathbf x}.
\label{eq:fourbesseloperator}
\end{equation}
%
Introducing the Fourier decomposition of $\tilde F_a$, \cref{eq:fmacmafourier}, in  \cref{eq:poissonotexact3}, we obtain
%
\begin{equation}
\begin{split}
    \epsilon_0 \nabla \cdot \mathbf E =& \sum_a q_a \int  dv_\parallel d\mu \frac{B_\parallel^*}{m}  \left( \Gamma_0[\lb F_a \rb_{\mathbf R}] +  2\pi\sum_{l=1}^{\infty} \frac{i^{l-1}}{l\Omega_a}\Gamma_l[C_{l a}+C_{-l a}]\right),
\end{split}
\label{eq:poissonpresqueexact}
\end{equation}
%
where the $\theta$ integration was performed by using the identity $\int_0^{2\pi} e^{i\theta(l-m)} d\theta = 2\pi \delta(l-m)$.
Notice that $\int_0^{2\pi}\Gamma_0[F_a]d\theta/2\pi = \Gamma_0(\lb F_a \rb_{\mathbf R})$, and corresponds to the $J_0(k_\perp \rho_a)$ operator used in most gyrofluid closures \citep{Hammett1992a,Dorland1993,Snyder2001,Madsen2013}, and in the gyrokinetic Poisson equation \citep{Lee1983,Dubin1983a}.

We now order the terms appearing in \cref{eq:poissonpresqueexact}. Using the Taylor series expansion of a Bessel function $J_l(x)$ of order $l$ \citep{Abramowitz1972}, we find
%
\begin{equation}
    \Gamma_0[\lb F_a \rb_{\mathbf R}] \sim \left[1 - \frac{(k_{\perp} \rho_a)^2}{4} + O(\epsilon^{4})\right]\lb F_a \rb_{\mathbf R},
\end{equation}
%
while using the orderings of $\nu_e$ and $\nu_i$ in \cref{eq:orderingnu,eq:orderingnu2}
%
\begin{equation}
    \frac{\Gamma_l[C_{la}]}{\Omega_a} < \epsilon_\nu\epsilon^{l+1} \lb F_a \rb_{\mathbf R}.
\end{equation}
%
for $l \ge 1$.
%
Consistently with \cref{section:cabmomentexpansion}, we neglect the $l\ge1$ collisional  terms, therefore representing Poisson's equation up to $O(\epsilon_\nu \epsilon)$.
%
Such terms are included in the gyrokinetic model in \cref{ch:gk}.
%
%For the derivation of an higher-order Poisson equation, the treatment of finite $l\ge1$ collisional effects are presented in Appendix \ref{app:poisson}.
%
Taylor expanding $J_0(x) \simeq 1-x^2/4$, Poisson's equation reads
%
\begin{equation}
\begin{split}
    \epsilon_0 \nabla \cdot \mathbf E=&\sum_a q_a\left[N_{a}\left(1+\frac{\mathbf b \cdot \nabla \times \mathbf b}{\Omega_a} u_{\parallel a}+\frac{\mathbf b \cdot \nabla \times \mathbf v_E}{\Omega_a}\right) +\frac{1}{2m_a}\nabla_\perp^2 \left(\frac{P_{\perp a}}{\Omega_a^2}\right)\right].
\end{split}
    \label{eq:poissonfin2}
\end{equation}


\section{Collisional Drift-Reduced Fluid Model}
\label{sec:fluidmodel}

The infinite set of equations that describe the evolution of the moments of the distribution function, \cref{eq:finalDKE}, and Poisson's equation, \cref{eq:poissonfin2}, constitute the drift-reduced model, which is valid for distribution functions arbitrarily far from equilibrium. For practical purposes, a closure scheme must be provided in order to reduce the model to a finite number of equations.
In this section, we derive a closure in the high collisionality regime. For this purpose, we first state in \cref{sec:fluideqs} the evolution equations for the fluid moments (i.e. $n_a, u_{\parallel a}, T_{\parallel a}, T_{\perp a}, Q_{\parallel a}$ and $Q_{\perp a}$), that correspond to the lowest-order indices of the moment-hierarchy equation. Then, in \cref{sec:highcoll}, we apply a prescription for the higher-order parallel and perpendicular moment equations that allows a collisional closure for $Q_{\parallel a}$ and $Q_{\perp a}$ in terms of $n_a, u_{\parallel a}, T_{\parallel a}$ and $T_{\perp a}$.
The nonlinear closure prescription used here, {sometimes called \textit{semi-collisional closure} \citep{Zocco2011}}, can be employed at arbitrary collisionalities by including a sufficiently high number of moments {[indeed, it was used in \citet{Zocco2015,Loureiro2015} to consider low collisionality regimes]. It also allows us to retain the non-linear collision contributions inherent to a full-F description that may have the same size as its linear contributions, as pointed out in \citet{Catto2004}.}

\subsection{Fluid Equations}
\label{sec:fluideqs}

We first look at the $(p,j)=(0,0)$ case of \cref{eq:finalDKE}. Noting that $C_{ab}^{00}=0$, we obtain
%
\begin{equation}
    \frac{\partial N_a^{*00}}{\partial t}+\nabla \cdot \left|\left|\dot{\mathbf R}\right|\right|^{*00}_a + \mathcal{F}_a^{00}=0.
    \label{eq:cont1}
\end{equation}
%
Evaluating $\left|\left|\dot{\mathbf R}\right|\right|^{*pj}_a$ in \cref{eq:finalDKE3} and $\mathcal{F}_a^{pj}$ in \cref{eq:finalDKEF}, for $(p,j)=(0,0)$, \cref{eq:cont1} yields the continuity equation
%
\begin{equation}
    \frac{d_a^0 N_a}{dt} + \frac{d_{0 a}}{dt}\left(\frac{N_a\nabla_\perp^2 \phi}{\Omega_a B}\right) = -N_a \nabla \cdot \mathbf u_{0 a} - \frac{N_a\nabla_\perp^2 \phi}{\Omega_a B}\nabla \cdot \mathbf U_{0 a}.
    \label{eq:continuity}
\end{equation}
%
The upper convective derivative ${d_a^0}/{dt}$, defined by
%
\begin{equation}
    \frac{d_a^0}{dt}=\frac{\partial}{\partial t} + \mathbf u_{0a} \cdot \nabla,
    \label{eq:convectop0}
\end{equation}
%
is related to the guiding-center fluid velocity $\mathbf u_{0a}$
%
\begin{equation}
    \mathbf u_{0a} = \mathbf U_{0a} + \frac{T_{\parallel a}+T_{\perp a}}{m_a}\frac{\mathbf b \times \nabla B}{\Omega_a B} +\frac{\mathbf b}{\Omega_a}\times \frac{d_{0 a} \mathbf U_{0 a}}{dt},
    \label{eq:guidvel}
\end{equation}
%
and it differs from the lower-convective derivative ${d_{0a}}/{dt}$ in \cref{eq:convdev0} by the addition of the last two terms in \cref{eq:guidvel}.
The vorticity $\nabla_\perp^2 \phi$ is related to the $\mathbf E \times \mathbf B$ drift by
%
\begin{equation}
    \frac{\mathbf b \cdot \nabla \times \mathbf v_E}{\Omega_a} = \frac{\nabla_\perp^2 \phi}{B \Omega_a} + O(\epsilon^3),
    \label{eq:vorticityapprox}
\end{equation}
%
and it appears in \cref{eq:continuity} due to the difference between $N_a^{*00}$ and $N_a^{00}$ [see \cref{eq:overlinenapj}].
To derive \cref{eq:continuity}, we use the low-$\beta$ limit expression for $\mathbf b \times \mathbf k \simeq (\mathbf b \times \nabla B)/B$ and neglect $u_{\parallel a}\mathbf b \cdot \nabla \times \mathbf b/\Omega_a$ as
%
\begin{equation}
    \frac{u_{\parallel a} \mathbf b \cdot \nabla \times \mathbf b}{\Omega_a} \sim \frac{T_e}{T_i}\beta \sim \epsilon^3,
    \label{eq:bcurlbapprox}
\end{equation}
%
therefore keeping up to $O(\epsilon^2)$ terms [namely the $\nabla_\perp^2 \phi$ term in \cref{eq:vorticityapprox}].
%
We note that, although the particle Lagrangian is kept up to $O(\epsilon)$, the Euler-Lagrange equations set the particle equations of motion and Botlzmann equation to be second order accurate in $\epsilon$.

The parallel momentum equation is obtained by setting $(p,j)=(1,0)$ in \cref{eq:finalDKE}, yielding
%
\begin{equation}
    \begin{split}
        m_a\frac{d_a^0 u_{\parallel a}}{dt} &=\frac{m_a v_{th\parallel a}}{\sqrt 2}\sum_b C_{ab}^{10}-\frac{m_a \nabla_\perp^2 \phi}{\Omega_a B}\frac{d_0 u_{\parallel a}}{dt} - \frac{m_a}{\sqrt 2N_a}\nabla \cdot\left(\mathbf u_a^1 N_a v_{th\parallel a} \right) \\
        &+m_a ||\mathcal{A}||_a^{*00}+\left(1+\frac{\nabla_\perp^2 \phi}{\Omega_a B}\right)\left(q_a E_\parallel - T_{\perp a} \frac{\nabla_\parallel B}{B}+m_a \mathbf v_E \cdot \frac{d_{0 a} \mathbf b}{dt}\right),
    \end{split}
    \label{eq:parallelvelgc}
\end{equation}
%
with 
%
\begin{equation}
    \begin{split}
        \mathbf u_a^1 &= \frac{\mathbf U_{p a}^{th}}{\sqrt{2}}+\frac{\sqrt 2}{m_a}\frac{\mathbf b \times \nabla B}{\Omega_a B}\frac{Q_{\parallel a}+Q_{\perp a}}{N_a v_{th\parallel a}}+v_{th\parallel a}\frac{\mathbf b}{2}\left(1+\frac{\nabla_\perp^2 \phi}{\Omega_a B}\right).
    \end{split}
\end{equation}
%
The expression for $C_{ab}^{10}$ is given in Appendix \ref{app:cabmoments}, as well as all the $C_{ab}^{pj}$ coefficients relevant for the present fluid model.
The left-hand side of \cref{eq:parallelvelgc} describes the convection of $u_{\parallel a}$, while the first term in the right-hand side is related to pressure and heat flux gradients, the second term to resistivity (collisional effects), the third term consists of high-order terms kept to ensure phase-space conservation properties, and the last term is the parallel fluid acceleration, namely due to parallel electric fields, mirror force, and inertia.

The parallel and perpendicular temperature equations are obtained by setting $(p,j)=(2,0)$ and $(0,1)$ respectively in \cref{eq:finalDKE}. This yields for the parallel temperature
%
\begin{equation}
    \begin{split}
        \frac{N_a}{\sqrt{2}}\frac{d_a^0 T_{\parallel a}}{dt} &=
        \sqrt{2} Q_{\perp a} \frac{\nabla_\parallel B}{B}- \frac{N_a \nabla_\perp^2 \phi}{\sqrt{2}\Omega_a B}\frac{d_{0a} T_{\parallel a}}{dt}-2\frac{N_a T_{\parallel a}}{v_{th\parallel a}}\mathbf u_a^{1} \cdot \nabla u_{\parallel a}\\
        &-\nabla \cdot (N_a T_{\parallel a} \mathbf u_a^{2\parallel})+ N_a T_{\parallel a}\frac{\mathbf E}{B}\cdot \frac{\mathbf b \times \nabla B}{B}\left(1+\frac{\nabla_\perp^2 \phi}{\Omega_a B}\right)\\
        &+\sum_b C_{ab}^{20}N T_{\parallel a} +\frac{2 N_a T_{\parallel a}}{v_{th\parallel a}}||\mathcal{A}||_a^{*10},
    \end{split}
\label{eq:paralleltempc}
\end{equation}
%
where
%
\begin{equation}
    \begin{split}
        \mathbf u_a^{2\parallel}&= \frac{Q_\parallel a}{2 N_a T_{\parallel a}}\frac{\mathbf U_{pa}^{th}}{v_{th\parallel a}}+\frac{\sqrt{2}T_{\parallel a}}{m_a}\frac{\mathbf b \times \nabla B}{\Omega_a B}+\frac{\mathbf b}{2}\frac{Q_{\parallel a} }{N_a T_{\parallel a} }\left(1+\frac{\nabla_\perp^2 \phi}{\Omega_a B}\right),
    \end{split}
\end{equation}
%
and for the perpendicular temperature
%
\begin{equation}
    \begin{split}
        &N_a\frac{d_a^0 }{dt}\left(\frac{T_{\perp a}}{B}\right)+\frac{N_a\nabla_\perp^2 \phi}{\Omega_a B}\frac{d_{0 a} }{dt}\left(\frac{T_{\perp a}}{B}\right) =\nabla \cdot \left(\frac{N_a T_{\perp a}}{B} \mathbf u_a^{2\perp}\right)-\frac{ N_a T_{\perp a}}{B}\sum_b  C_{ab}^{01},
    \end{split}
    \label{eq:perptempc}
\end{equation}
%
with
%
\begin{equation}
    \begin{split}
        \mathbf u_a^{2\perp}&=-\frac{Q_{\perp a}}{N_a T_{\perp a}}\frac{\mathbf U_{pa}^{th}}{v_{th\parallel a}}-\frac{T_{\perp a}}{m_a}\frac{\mathbf b \times \nabla B}{\Omega_a B}.
    \end{split}
\end{equation}
%
The equations for the evolution of the parallel $Q_{\parallel a}$ and perpendicular $Q_{\perp a}$ heat fluxes are obtained by setting $(p,j)=(3,0)$ and $(1,1)$ respectively in \cref{eq:finalDKE}, yielding
%
\begin{equation}
    \begin{split}
        \frac{d_a^0 Q_{\parallel a}}{dt} &=-  \frac{d_{0 a}}{dt}\left(Q_{\parallel a}\frac{\nabla_\perp^2 \phi}{\Omega_a B}\right)+N_a T_{\parallel a} \sqrt{3} v_{th\parallel a} \sum_b C_{ab}^{30} \\
        &-Q_{\parallel a}\nabla \cdot \mathbf u_a^0- \frac{Q_{\parallel a} \nabla_\perp^2 \phi}{\Omega_a B}\nabla \cdot \mathbf U_{0 a} - 3 \nabla \cdot (\mathbf u_{k a} Q_{\parallel a})\\
        &-\frac{3}{\sqrt{2}}\left(1+\frac{\nabla_\perp^2 \phi}{\Omega_a B}\right)\frac{\mathbf E \cdot \mathbf b \times \nabla B}{B^2}Q_{\parallel a}+3 \sqrt{2}N_a T_{\parallel a} ||\mathcal{A}||_a^{*20}\\
        &-3 \sqrt{2}N_a T_{\parallel a} \mathbf u_a^{2\parallel} \cdot \nabla u_{\parallel a} - 3 \sqrt{2}N_a v_{th\parallel a} \mathbf u_a^1 \cdot \nabla T_{\parallel a},
    \end{split}
    \label{eq:qpar}
\end{equation}
%
and
%
\begin{equation}
    \begin{split}
        \frac{d_a^0}{dt}\left(\frac{Q_\perp a}{B}\right)&=- \frac{d_{0 a}}{dt}\left(\frac{Q_{\perp a}}{B}\frac{\nabla_\perp^2 \phi}{\Omega_a B}\right)-\frac{N_a v_{th\parallel a}}{\sqrt{2}}(\mathbf u_a^1 \cdot \nabla) \frac{T_{\perp a}}{B}\\
        &+\frac{N_a T_{\perp a}}{B}(\mathbf u_a^{2\perp} \cdot \nabla) u_{\parallel a}-\left(\frac{Q_{\perp a}}{B}\right)\left(\nabla \cdot \mathbf u_a^0 + \frac{\nabla_\perp^2 \phi}{\Omega B}\nabla \cdot \mathbf U_{0 a}\right)\\
        &-(\mathbf U_{k a} + 2 \mathbf U_{\nabla B})\cdot \nabla \left(\frac{Q_{\perp a}}{B}\right)-\frac{\sum_b C_{ab}^{11}}{\sqrt{2}}\frac{v_{th\parallel a}N_a T_{\perp a}}{B}\\
        &+\left(\frac{N_a T_{\perp a}^2}{m_a}\frac{\nabla_\parallel B}{B^2}+\frac{Q_{\perp a}}{B}\mathbf E \cdot \frac{\mathbf b \times \nabla B}{B^2}\right)\left(1+\frac{\nabla_\perp^2 \phi}{\Omega_a B}\right).
    \end{split}
    \label{eq:qperp}
\end{equation}
%
{In \cref{eq:qpar,eq:qperp} we neglected the higher-order moments with respect to $N^{30}$ and $N^{11}$, an approximation that we will scrutinize in the next section.}
Equations (\ref{eq:continuity})-(\ref{eq:qperp}) constitute a closed set of six coupled non-linear partial differential equations for both the fluid variables $n_a, u_{\parallel a}, T_{\parallel a}, T_{\perp a}$, and the kinetic variables $Q_{\parallel a}$ and $Q_{\perp a}$.

With respect to previous $\delta$F \citep{Dorland1993,Brizard1994} and full-F gyrofluid models \citep{Madsen2013}, our fluid model, Eqs. (\ref{eq:continuity}-\ref{eq:qperp}), while neglecting $k_{\perp} \rho_i \sim 1$ effects, {includes the velocity contributions from} the $B_{\parallel}^*$ denominator in the equations of motion, \cref{eq:GC1,eq:GC2}, and includes the effects of full Coulomb collisions up to order $\epsilon_\nu \epsilon$.
{Also, due to the choice of basis functions with shifted velocity arguments $H_p(s_{\parallel a})$ instead of $H_p(v_\parallel/v_{tha})$, we obtain a set of equations that can efficiently describe both weak flow ($u_{\parallel a} \ll v_{th a})$ and strong flow ($u_{\parallel a} \sim v_{th a}$) regimes}.

\subsection{High Collisionality Regime}
\label{sec:highcoll}

We now consider the high collisionality regime, where the characteristic fluctuation frequency{, $\omega$,} of the {fluid} variables, {satisfies}
%
\begin{equation}
\begin{split}
	&\omega \sim v_{tha} |\nabla_\parallel \ln N_a| \sim v_{tha} |\nabla_\parallel \ln T_{\parallel a}|\sim v_{tha} |\nabla_\parallel \ln T_{\perp a}| \sim |\nabla_\parallel \ u_{\parallel a}| \sim v_{th a}/ L_{\parallel a},
\end{split}
\end{equation}
%
is much smaller than the collision frequency $\nu_a \simeq \nu_{aa}$, that is
%
\begin{equation}
    \delta_a \sim \frac{\omega}{\nu_a} \sim \frac{\lambda_{mfp a}}{L_{\parallel a}} \ll 1,
\label{eq:smallmfp}
\end{equation}
%
where the mean free path $\lambda_{mfp a}$ in \cref{eq:smallmfp} is defined as
%
\begin{equation}
    \lambda_{mfp a} = v_{th a}/\nu_{aa}.
\end{equation}
%
Equation (\ref{eq:smallmfp}) describes the so-called linear transport regime \citep{Balescu1988}.
In this case, the distribution function can be expanded around a Maxwell-Boltzmann equilibrium, according to the Chapman-Enskog asymptotic closure scheme \citep{Chapman1962} and, to first order in $\delta_a$, we have
%
\begin{equation}
    \lb F_a \rb_{\mathbf R} \simeq F_{Ma}\left[1+\delta_a f_{1 a}(\mathbf R,v_{\parallel},\mu,t)\right].
\label{eq:chapenskexpansion}
\end{equation}
%
According to \cref{eq:chapenskexpansion}, all moments $N_a^{pj}$ in the Hermite-Laguerre expansion \cref{eq:gyrof} with $(p,j)\not=(0,0)$ are order $\delta_a$. {Since $Q_{\parallel a}$ and $Q_{\perp a}$ are determined at first order in $\delta_a$ only by the moments $(p,j)=(0,0),(3,0),(1,1)$, the truncation of Sec. (\ref{sec:fluideqs}), i.e., neglecting $(p,j)\not=(0,0),(3,0),(1,1)$ is justified. For a more detailed discussion {on this topic} see \citet{Balescu1988}.
Moreover, in the linear regime, a relationship between the hydrodynamical and kinetic variables can be obtained along the lines of the semi-collisional closure. This allows us to express $Q_{\parallel a}$ and $Q_{\perp a}$ as a function of $N_a, u_{\parallel a}, T_{\parallel a}$ and $T_{\perp a}$, therefore reducing the number of equations.
We now derive this functional relationship.

We consider Eqs. (\ref{eq:qpar})-(\ref{eq:qperp}) in the linear regime, and neglect the polarization terms that are proportional to $\nabla_\perp^2 \phi/(\Omega_a B)$.
{This yields $\sqrt{{3}/{2}}{\sum_b C_{ab}^{30}}/{v_{th\parallel a}} \simeq R_{\parallel a}$ and ${\sum_b C_{ab}^{11}}/(\sqrt{{2}}{v_{th\parallel a}}) \simeq R_{\perp a}$, with $R_{\parallel a}$ and $R_{\perp a}$ given by}
%
\begin{equation}
    R_{\parallel a} = \frac{\nabla_{\parallel} T_{\parallel a}}{T_{\parallel a}} +u_{\parallel a} \frac{\mathbf b \times \nabla B}{\Omega_a B}\cdot \left(\frac{\nabla u_{\parallel a}}{u_{\parallel a}}+\frac{\nabla T_{\parallel a}}{T_{\parallel a}}\right),
    \label{eq:collimit1}
\end{equation}
\begin{equation}
\begin{split}
    R_{\perp a} &= \frac{T_{\perp a}}{T_{\parallel a}}\frac{\nabla_{\parallel} B}{B} - \frac{1}{2\sqrt{2}}\nabla_{\parallel} \ln \frac{T_{\perp a}}{B}-u_{\parallel a} \frac{\mathbf b \times \nabla B}{\Omega_a B}\cdot \left(\frac{T_{\perp a}}{T_{\parallel a}}\frac{\nabla u_{\parallel a}}{u_{\parallel a}}+\nabla \ln \frac{T_{\perp a}}{B}\right),
\end{split}
\label{eq:collimit2}
\end{equation}
%
since $d_a^0/dt \sim d_{0 a}/dt \sim \omega{}$ and $(d^0 Q_{\parallel, \perp }/dt)/Q_{\parallel,\perp a} \sim \delta_a^2 \nu_a$.
We compute the guiding-center moments of the collision operator $C_{ab}^{30}$ and $C_{ab}^{11}$ by truncating the series for the like-species collision operator in \cref{eq:caapjexact} at $(l,k,n,q)=(2,1,2,1)$.
The resulting $C_{ab}^{pj}$ coefficients are presented in Appendix \ref{app:cabmoments}.

With the expression of $C_{ab}^{30}$ and $C_{ab}^{11}$, we can solve for $Q_{\parallel a}$ and $Q_{\perp a}$. In the regime $(T_{\parallel a}-T_{\perp a})/T_a \sim \delta$, at lowest order, we obtain for the electron species
%
\begin{equation}
    \frac{Q_{\parallel e}}{N_e T_{ e} v_{th e}} =-0.362 \frac{u_{\parallel e}-u_{\parallel i}}{v_{th e}}-10.6 \lambda_{mfpe}\frac{\nabla_\parallel T_e}{T_e},
    \label{eq:qpare}
\end{equation}
%
and
%
\begin{equation}
    \frac{Q_{\perp e}}{N_e T_{ e} v_{th e}} =-0.119 \frac{u_{\parallel e}-u_{\parallel i}}{v_{th e}}-3.02 \lambda_{mfpe}\frac{\nabla_\parallel T_e}{T_e}.
\label{eq:qperpe}
\end{equation}
%
Analogous expressions are obtained for the ion species.

Equations (\ref{eq:continuity}), (\ref{eq:parallelvelgc}), (\ref{eq:paralleltempc}), and (\ref{eq:perptempc}), with $Q_{\parallel a}$ and $Q_{\perp a}$ given by \cref{eq:qperpe,eq:qpare} are valid in the high collisionality regime, and can be compared with the drift-reduced Braginskii equations in \citet{Zeiler1997}. We first rewrite the continuity equation, \cref{eq:continuity}, in the form
\begin{equation}
    \frac{\partial N_e}{\partial t} + \nabla \cdot \left[N_e\left(\mathbf v_E + u_{\parallel e} \mathbf b+  \frac{T_{\parallel e}+T_{\perp e}}{m_e}\frac{\mathbf b \times \nabla B}{\Omega_e B} \right)\right]
    =0,
\end{equation}
\noindent where we expand the convective derivative $d^0{a}/dt$ using \cref{eq:convectop0} and \cref{eq:guidvel}, and neglect polarization terms proportional to the electron mass $m_e$. By noting that the diamagnetic drift $v_{de}$ can be written as 
\begin{equation}
    \mathbf v_{de} = \frac{1}{e N_e}\nabla \times \frac{p_e \mathbf b}{B}-2\frac{T_e}{m_e}\frac{\mathbf b \times \nabla B}{\Omega_e B},
\end{equation}
\noindent and by considering the isotropic regime $T_{\parallel e} \sim T_{\perp e} \sim T_{e}$, we obtain
\begin{equation}
    \frac{\partial N_e}{\partial t} + \nabla \cdot \left[N_e\left(\mathbf v_E + u_{\parallel e} \mathbf b+  \mathbf v_{de} \right)\right]
    =0,
\end{equation}
\noindent which corresponds to the continuity equation in the drift-reduced Braginskii model in \citet{Zeiler1997}. In {that model}, the polarization equation is obtained by subtracting both electron and ion continuity equations, using Poisson's equation $n_e \simeq n_i$ with $n_e$ and $n_i$ the electron and ion particle densities respectively, and neglecting {terms proportional to} the electron to ion mass ratio. Applying the same procedure to the present fluid model, we obtain
%
\begin{equation}
\begin{split}
    0&=\nabla \cdot \left(\frac{\nabla_\perp^2 \phi N_i u_{\parallel i} \mathbf b}{\Omega_i B}\right)-\nabla \cdot \left[\frac{\mathbf v_E}{2 m_i}\nabla_\perp^2\left(\frac{N_i T_{\perp i}}{\Omega_i^2}\right)\right]-\frac{1}{2m_i}\frac{\partial}{\partial t}\nabla_\perp^2 \left(\frac{N_i T_{\perp i}}{\Omega_i^2}\right)\\
    &+\nabla \cdot \left(\frac{N_i}{\Omega_i}\mathbf b \times \frac{d_{0i}\mathbf U_{0i}}{dt}\right)+\nabla \cdot \left[\mathbf b\left(N_i u_{\parallel i} - N_e u_{\parallel e}\right)\right]\\
    &+\nabla \cdot \left[\left({N_i T_{\parallel i}+N_eT_{\parallel e}+N_i T_{\perp i}+N_eT_{\perp e}}\right)\frac{\mathbf b \times \nabla B}{e B^2}\right].
\end{split}
\label{eq:gcpolarizationeq}
\end{equation}

In \cref{eq:gcpolarizationeq}, the first three terms, which are not present in the drift-reduced Braginskii model, correspond to the difference between ion guiding-center density $N_i$ and particle density $n_i$, proportional to both $\nabla_\perp^2 \phi$ and $\nabla_\perp^2 P_i$.
The parallel momentum and temperature equations, \cref{eq:parallelvelgc} and \cref{eq:paralleltempc}, with respect to \citep{Zeiler1997}, contain the higher-order term $\mathcal{A}{\sim O(\epsilon^2)}$  that ensures phase-space conservation, mirror force terms proportional to $(\nabla_\parallel B)/B$, and polarization terms proportional to $\nabla_\perp^2 \phi/(\Omega_a B)$ due to the difference between guiding-center and particle fluid quantities.
{This set of fluid equations constitute an improvement over the drift-reduced Braginskii model. With respect to the original Braginskii equations \citep{Braginskii1965}, they include the non-linear terms that arise when retaining full Coulomb collisions, and the effect of ion-electron collisions.}

\section{Conclusion}

In this chapter, a drift-kinetic model is developed, suitable to describe the plasma dynamics in the SOL region of tokamak devices at arbitrary collisionality. Taking advantage of the separation between the turbulent and gyromotion scales, a gyroaveraged Lagrangian and its corresponding equations of motion are obtained. This is the starting point to deduce a drift-kinetic Boltzmann equation with full Coulomb collisions for the gyroaveraged distribution function.

The gyroaveraged distribution function is then expanded into an Hermite-Laguerre basis, and the coefficients of the expansion are related to the lowest-order gyrofluid moments. The fluid moment expansion of the Coulomb operator described in \citet{Ji2009} is reviewed, and its respective particle moments are written in terms of coefficients of the Hermite-Laguerre expansion, relating both expansions. This allows us to express analytically the moments of the collision operator in terms of guiding-center moments.
A moment-hierarchy that describes the evolution of the guiding-center moments is derived, together with a Poisson's equation accurate up to $\epsilon^2$. These are then used to derive a fluid model in the high collisionality limit.

The drift-kinetic model derived herein {will be considered in \cref{ch:gk} as} a starting point for the development of a gyrokinetic Boltzmann equation suitable for the SOL region (e.g. \citet{Qin2007, Hahm2009}). Indeed, using a similar approach, a gyrokinetic moment-hierarchy may be derived, allowing for the use of perpendicular wave numbers satisfying $k_\perp \rho_s \sim 1$.