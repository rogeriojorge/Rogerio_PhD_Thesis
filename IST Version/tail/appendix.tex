\appendix
\chapter{Drift-Kinetic Basis Transformation}
\label{app:tlkpj}

In the present Appendix, we derive the expressions for the coefficients $T_{alk}^{pj}$ appearing in \cref{eq:tlkpj}. These coefficients allows us to express up to order $\epsilon \epsilon_\nu$ the relation between fluid $\bm M_{a}^{lk}$ and guiding-center $N_{a}^{lk}$ moments via \cref{eq:CoulDKmom}.
As a first step, we define a transformation similar to \cref{eq:tlkpj} but with isotropic temperatures between both {bases}

\be
    \begin{split}
        c_a^l P_l(\xi_a)L_k^{l+1/2}(c_a^2)=&\sum_{p=0}^{l+2k}\sum_{j=0}^{k+\floor{l/2}}\overline T_{lka}^{pj} H_p\left(\frac{v_\parallel-u_{\parallel a}}{v_{th a}}\right) L_j\left(\frac{v^{'2}_{\perp}}{v_{th a}^2}\right),
    \end{split}
    \label{eq:deftbarlkpj}
\ee

\noindent with the inverse transformation

\be
    \begin{split}
        H_p\left(\frac{v_\parallel-u_{\parallel a}}{v_{th a}}\right)L_j\left(\frac{v^{'2}_{\perp}}{v_{th a}^2}\right) =& \sum_{l=0}^{p+2j}\sum_{k=0}^{j+\floor{p/2}}\left(\overline T^{-1}\right)_{pj a}^{lk} \\
        &\times c_a^l P_l(\xi_a)L_k^{l+1/2}(c_a^2),
    \end{split}
\ee

The relation between the coefficients $\left(\overline T^{-1}\right)_{pj}^{lk}$ and $\overline T_{lk}^{pj}$ is given by

\be
    \left(\overline T^{-1}\right)_{pj}^{lk}=\frac{\sqrt{\pi}2^p p!(l+1/2)k!}{(k+l+1/2)!}\overline T_{lk}^{pj}.
\ee

By integrating both sides of \cref{eq:deftbarlkpj} over the whole velocity space, we can write $\overline{T}_{lk}^{pj}$ as
%
\begin{align}
    T_{lk}^{pj}=\int P_l(\xi)c^l L_k^{l+1/2}(c^2)\frac{H_p(s_\parallel) L_j(s_\perp^2)}{2^p p! \sqrt{\pi}}e^{-s_\parallel^2-s_\perp^2}ds_\parallel ds_\perp^2,
\label{eq:deftlkpj1}
\end{align}
%
where we suppressed the species index $a$ for simplicity, and find
%
\begin{equation}
    \begin{split}
        \overline T_{lk}^{pj}&=\sum_{q=0}^{\floor{l/2}}\sum_{v=0}^{\floor{p/2}}\sum_{i=0}^{k}\sum_{r=0}^{q}\sum_{s=0}^{\text{min}(j,i)}\sum_{m=0}^{k-i}\frac{(-1)^{q+i+j+v+m}}{2^{\frac{3l+p}{2}+m+v-r}}\\
        &\times\binom{l}{q}\binom{2(l-q)}{l}\binom{q}{r}\binom{r}{j-s}\binom{r}{i-s}\binom{s+r}{s}{r!}{}\\
        &\times\frac{(k-i+l-1/2)!(l+p+2(m-r-v)-1)!!}{(p-2v)!(k-i-m)!(l+m-1/2)!v!m!}.
    \end{split}
\end{equation}

We then integrate both sides of \cref{eq:tlkpj} with weights $H_l(s_{\parallel a})L_j(s_{\perp a}^2)$, with the argument transformation

\begin{equation}
\begin{split}
        H_p(s_{\parallel a})=\left(\frac{T_a}{T_{\parallel a}}\right)^{p/2}\sum_{k=0}^{\floor{p/2}}&\frac{p!}{k!(p-2k)!}\left(1-\frac{T_{\parallel a}}{T_a}\right)^kH_{p-2k}\left(\frac{v_\parallel-u_{\parallel a}}{v_{th a}}\right),
\end{split}
\end{equation}

\noindent and

\begin{equation}
\begin{split}
    L_j(s_{\perp a}^2)=\sum_{k=0}^{j}&\binom{j}{j-k}\left(\frac{T_a}{T_{\perp a}}\right)^k \left(1-\frac{T_a}{T_{\perp a}}\right)^{j-k}L_{k}\left(\frac{v_{\perp}^{'2}}{v_{th a}^2}\right),
\end{split}
\end{equation}

\noindent to find the relation between the isotropic and anisotropic temperature coefficients

\be
    \begin{split}
        T_{alk}^{pj}=&\sum_{m=0}^{l+2k}\sum_{n=0}^{k+\floor{l}{2}}\sum_{z=0}^{n}\sum_{d=0}^{\floor{m/2}}\binom{n}{n-z}\frac{m!\delta_{z,j}\delta_{p,m-2d}}{d!(m-2d)!}\\
        &\times\left(\frac{T_{\parallel a}}{T_a}\right)^{p/2}\left(\frac{T_{\perp a}}{T_a}\right)^{z}\left(1-\frac{T_a}{T_{\parallel a}}\right)^{d}\left(1-\frac{T_{\perp a}}{T_a}\right)^{n-z}{\overline{T}_{lk}^{m n}},
    \end{split}
    \label{eq:tlkpjexact}
\ee

\be
    \begin{split}
        \left(T^{-1}_a\right)_{pj }^{lk}=&\sum_{z=0}^{j}\sum_{d=0}^{\floor{p/2}}\sum_{t=0}^{p-2d+2z}\sum_{v=0}^{z-d+\floor{p}{2}}\binom{j}{j-z}\frac{p!\delta_{l,t}\delta_{k,v}}{d!(p-2d)!}\\
        &\times\left(\frac{T_{ a}}{T_{\parallel a}}\right)^{p/2}\left(\frac{T_{a}}{T_{\perp a}}\right)^{z}\left(1-\frac{T_{\parallel a}}{T_{ a}}\right)^{d}\left(1-\frac{T_{ a}}{T_{\perp a}}\right)^{j-z}\left(\overline{T^{-1}}\right)_{p-2d z}^{t v}.
    \end{split}
    \label{eq:tlkpjminus1exact}
\ee

A more efficient algorithm can be found as follows.
%
First, we expand the product $P_l(\xi)c^l L_k^{l+1/2}(c^2)$ into products of $s_\parallel$ and $s_\perp^2$ in order to write \cref{eq:deftlkpj1} in terms of $s_\parallel$ and $s_\perp^2$ only
%
\begin{align}
    P_l(\xi)c^l L_k^{l+1/2}(c^2)&=\sum_{i=0}^{\floor{l/2}}\sum_{m=0}^k\sum_{r=0}^{m+i}\binom{2l-2i}{l}\binom{l}{i}\binom{m+i}{r}\frac{(-1)^{i+m}(l+k+1/2)!)}{2^l(k-m)!(l+m+1/2)!m!}\nonumber\\
    &\times\frac{s_\parallel^{l-2i+2r}s_\perp^{2(m+i-r)}}{(T_\parallel/T)^{l/2-i+r}(T_\perp/T)}.
\end{align}
%
We then perform the parallel and perpendicular integrations separately, using the fact that
%
\begin{equation}
    \int_{-\infty}^{\infty} x^n \frac{H_p(x)}{2^p p! \sqrt{\pi}}e^{-x^2}dx=\frac{n!}{2^n}\frac{1-\mod(n-p,2)}{\left(\frac{n-p}{2}\right)!p!},
\label{eq:inthp1}
\end{equation}
%
and
%
\begin{equation}
        \int_0^{\infty} x^m L_j(x) e^{-x}dx = m!\binom{m}{m-j}(-1)^j.
\label{eq:intlj1}
\end{equation}
%
Finally, we apply \cref{eq:inthp1,eq:intlj1} to \cref{eq:deftlkpj1}, yielding
%
\begin{align}
    T_{lk}^{pj}&=\sum_{i=0}^{\floor{l/2}}\sum_{m=0}^k\sum_{r=0}^{m+i}\binom{2l-2i}{l}\binom{l}{i}\binom{m+i}{r}\frac{(-1)^{i+m+j}(l+k+1/2)!}{2^l(k-m)!(l+m+1/2)!m!}\nonumber\\
    &\times\binom{m+i-r}{m+i-r-j}\frac{(l-2i+2r)!}{2^{l-2i+2r}}\frac{1-\mod(l-p,2)}{\left(\frac{l-p}{2}-i+r\right)!p!}(m+i-r)!.
\end{align}

\chapter{Expressions for the Moments of the Collision Operator}
\label{app:cabmoments}

In the present Appendix, we present the expressions for the guiding-center moments of the collision operator relevant for the fluid model in \cref{sec:fluidmodel}.
The collision operator moments satisfy particle conservation
\be
    C_{ab}^{00}=0,
\ee

\noindent and momentum conservation {at lowest order}

\be
    C_{aa}^{10}=0,
\ee

\begin{equation}
    C_{ei}^{10}=-\frac{m_i}{m_e}\frac{v_{th\parallel i}}{v_{th \parallel e}}C_{ie}^{10}{+O({m_e/m_i})}.
\end{equation}

\noindent Both the like-species and electron-ion satisfy energy conservation exactly, {while the ion-electron operator satisfies \cref{eq:cabenergy} at zeroth order in $\delta_a$}

\begin{equation}
    T_{\parallel a}C_{ab}^{20}-\sqrt{2}T_{\perp a} C_{ab}^{01} = 0.
    \label{eq:cabenergy}
\end{equation}

{The remaining moments $C_{ab}^{pj}$, in the linear transport regime with $\Delta T_a/T_a = (T_{\parallel a} - T_{\perp a})T_a \sim N^{11}\sim N^{30} \sim (u_{\parallel e}-u_{\parallel i})/v_{the} \sim \delta_a$, for ion-electron collisions are given by}

\begin{align}
    C_{ie}^{10}&=-\frac{m_e}{m_i}\frac{v_{th \parallel i}}{v_{th \parallel e}}C_{ei}^{10},\\
    C_{ie}^{20}&=\sqrt{2}\nu_{ei}\frac{m_e}{m_i}\left(\frac{T_e-T_i}{T_i}\right)-\frac{2 \sqrt{2} \nu_{ei}}{3}\frac{m_e}{m_i}\frac{T_e}{T_i}\frac{\Delta T_i}{T_i},\\
    C_{ie}^{01}&=-2\nu_{ei}\frac{m_e}{m_i}\left(\frac{T_e-T_i}{T_i}\right)-\frac{2 \nu_{ei}}{3}\frac{m_e}{m_i}\frac{T_e}{T_i}\frac{\Delta T_i}{T_i},\\
    C_{ie}^{30}&=-\nu_{ei}\sqrt{\frac{3}{2}}\frac{m_e}{m_i}\frac{Q_{\parallel i}}{n T_i v_{thi}},\\
    C_{ie}^{11}&=3 \nu_{ei}\frac{m_e}{m_i}\frac{Q_{\perp i}}{n T_i v_{thi}},
\end{align}

\noindent {for electron-ion collisions}

{
\begin{align}
    C_{ei}^{10} &= -\frac{\sqrt{2}\nu_{ei}}{6 \pi^{3/2}}\frac{u_{\parallel e}-u_{\parallel i}}{v_{the}}+\frac{\sqrt{2}\nu_{ei}}{10 \pi^{3/2}}\frac{Q_{\parallel e}+2 Q_{\perp e}}{n T_e v_{the}},\\
    C_{ei}^{20}&=-\frac{2 \sqrt{2}\nu_{ei}}{15 \pi^{3/2}}\frac{\Delta T_e}{T_e},\\
    C_{ei}^{30}&= \frac{\sqrt{3}\nu_{ei}}{10 \pi^{3/2}}\frac{u_{\parallel e}-u_{\parallel i}}{v_{the}}-\frac{ \nu_{ei}}{70 \sqrt{3} \pi^{3/2}}\frac{31 Q_{\parallel e} - 2 Q_{\perp e}}{n T_e v_{the}},\\
    C_{ei}^{11}&= \frac{ \nu_{ei}}{5 \sqrt{2} \pi^{3/2}}\frac{u_{\parallel e}-u_{\parallel i}}{v_{the}}+\frac{ \nu_{ei}}{150 \sqrt{2} \pi^{3/2}}\frac{Q_{\parallel e}-94 Q_{\perp e}}{n T_e v_{the}},\\
\end{align}
}
\noindent {and for like-species collisions}
{
\begin{align}
    C_{aa}^{20}&=0,\\
    C_{aa}^{30}&=-\frac{2 \sqrt{2}}{125 \sqrt{3} \pi^{3/2}}\frac{\nu_{aa}}{n T_a v_{tha}}\left(19 Q_{\parallel a}-7 Q_{\perp a}\right),\\
    C_{aa}^{11}&=-\frac{2}{375 \pi^{3/2}}\frac{\nu_{aa}}{n T_a v_{tha}}\left(7 Q_{\parallel a}-121 Q_{\perp a}\right).
\end{align}
}


%%%%%%%%%%%%%%%%%%%%%%%%%%%%%%%%%%%%%%%%%%%%%%%%%%%%%%%%%%%%%%
\chapter{Spherical Basis Tensors}
\label{app:basistensors}

We start with the definition of the $\mathbf{Y}^l(\mathbf v)$ tensor in terms of spherical basis tensors $\mathbf e^{lm}$ in \cref{eq:ylvfull}.
%
For the $l=1$ case, \cref{eq:ylvfull} yields
%
\begin{equation}
    \mathbf Y^1(\mathbf v) = \mathbf v = \sqrt{\frac{4 \pi}{3}}v \sum_{m=-1}^1 Y_{1m}(\phi,\theta) \mathbf e^{1m}.
\label{eq:Y1defe}
\end{equation}
%
The spherical basis vectors $\mathbf e^{1m}$ can then be derived from \cref{eq:Y1defe} by decomposing the vector $\mathbf v$ in spherical coordinates as
%
\begin{equation}
    \mathbf v = v\left(\sin \phi \cos \theta \mathbf e_x+\sin \phi \sin \theta \mathbf e_y+\cos \phi \mathbf e_z\right),
\end{equation}
%
and using the identities for the spherical harmonics
%
\begin{equation}
    Y_{1m}(\phi,\theta)=
\begin{cases}
    \sqrt{\frac{3}{8 \pi}}\sin \phi e^{-i \theta}, &m=-1,\\
    \sqrt{\frac{3}{4 \pi}}\cos \phi, &m=0,\\
    -\sqrt{\frac{3}{8 \pi}}\sin \phi e^{i \theta}, &m=1,\\
\end{cases}
\end{equation}
%
therefore obtaining
%
\begin{equation}
    \mathbf e^{1m}=
\begin{cases}
    \frac{\mathbf e_x-i \mathbf e_y}{\sqrt{2}}, &m=-1,\\
    \mathbf e_z, &m=0,\\
    -\frac{\mathbf e_x+i \mathbf e_y}{\sqrt{2}}, &m=1.\\
\end{cases}
\end{equation}

We now construct spherical basis tensors $\mathbf e^{lm}$ from the spherical basis vectors $\mathbf e^{1m}$ leveraging the techniques developed for the angular momentum formalism in quantum mechanics \citep{Zettili2009,Snider2018}.
%
Indeed, the basis vectors $\mathbf e^{1m}$ are eigenvectors of the angular momentum matrix $G_z$
%
\begin{equation}
    G_z=i
  \begin{pmatrix}
    0 & -1 & 0\\
    1 &  0 & 0\\
    0 &  0 & 0
  \end{pmatrix},
\end{equation}
%
with eigenvalue $m$, that is
%
\begin{equation}
    G_z \cdot \mathbf e^{1m} = m \mathbf e^{1m}.
\end{equation}
%
In general, the angular momentum matrices along any axis $n={x,y,z}$ are given by
%
\begin{equation}
    G_{n} = -i \mathbf e_{n} \cdot \epsilon,
\label{eq:gnmatr}
\end{equation}
%
with $\epsilon$ the standard Levi-Civita tensor.
%
In index notation, \cref{eq:gnmatr} can be written as
%
\begin{equation}
    \left({G_{n}}\right)_{kl}=-i\sum_{j=1}^3\left(e_{n}\right)_j \epsilon_{jkl}.
\end{equation}
%
The raising $G_+$ and lowering $G_-$ operators (corresponding to the ladder operators in quantum mechanics) are defined by
%
\begin{equation}
    G_{\pm}=G_x \pm i G_y.
\end{equation}
%
The allow us to obtain the basis vectors $\mathbf e^{1\pm1}$ from $\mathbf e^{10}$ using
%
\begin{equation}
    G_{\pm}\mathbf e^{0m}=\mathbf e^{1\pm}.
\end{equation}
%
Finally, we note that  the dual basis $\mathbf e^{1}_m = (\mathbf e^{1}_m)^* = (-1)^m \mathbf e^{1-m}$, together with $\mathbf e^{1m}$, satisfy
%
\begin{equation}
    \mathbf e^{1m} \cdot \mathbf e^{1}_{m'} = \delta_{m,m'}.
\label{eq:orthoe}
\end{equation}

To obtain the spherical tensor basis $\mathbf e^{l m}$ for the irreducible tensors $\mathbf Y^{l}$, we start with the spherical basis tensor
%
\begin{equation}
    \mathbf e^{ll}=\mathbf e^{11}\mathbf e^{11}...\mathbf e^{11},
\end{equation}
%
formed by the product of $l$ basis vectors $\mathbf e^{11}$.
%
Indeed, similarly to $\mathbf Y^l(\mathbf v)$, this tensor is of rank $l$, symmetric, and traceless between any of its indices, as $\mathbf e^{11}\cdot \mathbf e^{11} = 0$.
%
Furthermore, we note that $\mathbf e^{ll}$ is an eigenvector with eigenvalue $l$ of the angular momentum tensor $G_z^{l}$, with $G_n^{l}$ defined by
%
\begin{equation}
    \left[G_n^{l} \cdot T^l\right]_{j k ... l}=\sum_{j'k'...l'}\left\{\left[G_n\right]_{jj'}\delta_{kk'}...\delta_{ll'}+\left[ \delta_{jj'}G_n\right]_{kk'}...\delta_{ll'}+...+\delta_{jj'}\delta_{kk'}...\left[G_n\right]_{ll'}\right\} T^l_{j'k'...l'},
\end{equation}
%
where $T^l$ is an arbitrary tensor of rank $l$.
%
The remaining basis tensor elements $\mathbf e^{l m}$ can be obtained by applying the tensorial lowering operator $G^l_-=G_x^{l}-i G_y^{l}$ to $\mathbf e^{ll}$, namely
%
\begin{equation}
    \mathbf e^{l m} = \sqrt{\frac{(l+m)!}{(2l)!(l-m)!}}G^{l}_-\cdot^{l-m}\mathbf e^{ll},
\label{eq:elmbasistensor}
\end{equation}
%
with $m=-l,-l+1,...,-1,0,1,...,l$.
%
The normalization factor in \cref{eq:elmbasistensor} is obtained by requiring that the contravariant $\mathbf e^{lm}$ and the covariant $\mathbf e^l_m=(\mathbf e^l_m)^*$ basis tensors form an orthonormal basis, i.e.,
%
\begin{equation}
    \mathbf e^{l m} \cdot \mathbf e^{l}_{m'} = \delta_{m,m'}.
\end{equation}
%
For computational purposes, we note that the tensor $\mathbf e^{l m}$ can also be written as a function of the basis vectors $\mathbf e^{1m}$ as \citep{Snider2018}
%
\begin{equation}
    \mathbf e^{l m}=N_{lm}\sum_{n=0}^{\floor{\frac{l+m}{2}}}a_n^{lm}\left\{(\mathbf e^{11})^{m+n}(\mathbf e^{1-1})^{n}(\mathbf e^{10})^{l-m-2n}\right\}_{TS},
\end{equation}
%
where $N_{lm}=\sqrt{(l+m)!(l-m)!2^{l-m}/(2l)!}$ and $a_n^{lm}=l!/[2^n n!(m+n)!(l-m-2n)!]$.

%%%%%%%%%%%%%%%%%%%%%%%%%%%%%%%%%%%%%%%%%%%%%%%%%%%%%%%%%%%%%%
\chapter{Gyrokinetic Basis Transformation}
\label{app:tlkpj1}

In this section, we derive a closed form expression for the $T_{lkm}^{pj}$ and $(T^{-1})^{lkm}_{pj}$ coefficients defined in \cref{eq:plljhplj,eq:plljhpljminus1}.
%
By multiplying \cref{eq:plljhplj} by a Hermite and a Laguerre polynomial and by an exponential of the form $e^{-\overline v^2}$, and integrating over the whole $\overline v_\parallel$ and $\overline \mu$ space, we obtain the following integral expression for $T_{lkm}^{pj}$
%
\begin{equation}
    T_{lkm}^{pj}=\frac{v_{tha}^{m-l}}{2^p p! \sqrt{\pi}} \int \frac{\overline v^l}{\overline v_\perp^m} P_l^m\left(\frac{\overline v_\parallel}{\overline v}\right) L_k^{l+1/2}\left(\frac{\overline v_{tha}^m}{\overline v_\perp^m}\right) H_p\left(\frac{\overline v_{\parallel a}}{v_{tha}}\right)L_j\left(\frac{\overline v_\perp^2}{v_{tha}^2}\right) e^{-\frac{v^2}{v_{tha}^2}}\frac{d \bm v}{2 \pi}.
\label{eq:appbas1}
\end{equation}
%
We first write the integrand in \cref{eq:appbas1} in terms of $\overline \xi=\overline v_\parallel/\overline v$ and $\overline v$ coordinates using the basis transformation in \cref{eq:plljhpljminus1}, yielding
%
\begin{equation}
\begin{split}
     T_{lkm}^{pj}&= \sum_{l'=0}^{p+2j}\sum_{k'=0}^{j+\floor{p/2}}\frac{(l+1/2)k!}{(l+k+1/2)!}T_{l'k'}^{pj}\\
     &\times\int_{-1}^1 \frac{P_l^m(\overline \xi) P_{l'}(\overline \xi)}{(1-\overline \xi)^2} d \overline \xi \int_0^\infty x_a^{(l+l'-m+1)/2}L_k^{l+1/2}(x_a)L_{k'}^{l'+1/2}(x_a) dx_a,
\end{split}
\label{eq:appbas2}
\end{equation}
%
where we used the fact that $(T^{-1})_{lk}^{pj}=T_{lk}^{pj}\sqrt{\pi}2^p p! k! (l+1/2)/(k+l+1/2)!$ \citep{Jorge2017}.
%
The first integral in \cref{eq:appbas2} is performed by expanding $P_{l}$ as a finite sum of the form
%
\begin{equation}
    P_{l}(x)=\sum_{s=0}^l c_s^l x^s,
\end{equation}
%
with the coefficients $c_s^l=2^l[(l+s-1)/2]!/[s!(l-s)!((s-l-1)/2)!]$, and using the relation between associated Legendre functions $P_l^m(x)$ and Legendre polynomials $P_l(x)$
%
\begin{equation}
    P_l^m(x) = (-1)^m(1-x^2)^{m/2}\frac{d^m P_l(x)}{dx^m}.
\end{equation}
%
The second integral in \cref{eq:appbas2} is performed by using the expansion of the associated Laguerre polynomials in \cref{eq:asslaguerre}.
%
The $T_{lkm}^{pj}$ coefficient can then be written as
%
\begin{align}
    T_{lkm}^{pj}&=\sum_{l'=0}^{p+2j}\sum_{k'=0}^{j+\floor{p/2}} T_{l'k'}^{pj}\frac{(l'+1/2)k'!}{(l'+k'+1/2)!} \sum_{m_1=0}^k\sum_{m_2=0}^{k'}\sum_{s_1=m}^l\sum_{s_2=0}^{l'}L_{k m_1}^l L_{k' m_2}^{l'}\nonumber\\
    &\times \frac{c_{s_1}^l c_{s_2}^{l'}}{2} \frac{s_1!}{(s_1-m)!} \frac{\left[1+(-1)^{s_1+s_2-m}\right]}{s_1+s_2+1-m}\left(m_1+m_2+\frac{l+l'-m+1}{2}\right)!.
\end{align}
%
The inverse transformation coefficients $(T^{-1})^{lkm}_{pj}$ defined by \cref{eq:plljhpljminus1} can be found similarly, yielding
%
\begin{equation}
    (T^{-1})^{lkm}_{pj}=\frac{2^p p! \sqrt{\pi} k! (l+1/2)(l-m)!}{(k+l+1/2)!(l+m)!}T_{lkm}^{pj}.
\end{equation}