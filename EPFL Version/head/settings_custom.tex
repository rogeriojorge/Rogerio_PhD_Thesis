%%%%%%%%%%%%%%%%%%%%%%%%%%%%%%%%%%%%%%%%%%%%%%
%
%		Thesis Settings
%		Custom settings
%
%%%%%%%%%%%%%%%%%%%%%%%%%%%%%%%%%%%%%%%%%%%%%%

%
%   Use this file for your own custom packages, command-definitions, etc...
%

% the following lines are for creating a simplified TO-DO box. However since boites is not per default installed with all latex-distributions, we have removed this example again
% if you want to use it and do not have "boites" installed, you can get it from here: http://www.ctan.org/tex-archive/macros/latex/contrib/boites
%
\usepackage{boites,boites_exemples}
\newcommand{\todolist}[1]{\begin{boiteepaisseavecuntitre}{TO DO in this chapter} #1 \end{boiteepaisseavecuntitre}}  % creates a little box
% \newcommand{\todolist}[1]{}  % to be used when to do is not to be printed

% \numberwithin{equation}{section}  % reset equation counters at start of each "section" and prefix numbers by section number
% \numberwithin{figure}{section}    % reset figure   counters at start of each "section" and prefix numbers by section number

\usepackage{mathtools}
\usepackage{acronym}
\usepackage{amsmath,amssymb,amsbsy}
\usepackage{bbm,bm,mathrsfs,yfonts}
\usepackage[capitalize]{cleveref}
\usepackage{float}
\usepackage{relsize}
\usepackage{datetime}
\newdateformat{monthyeardate}{%
  \monthname[\THEMONTH] \THEYEAR}

\newcommand{\be}{\begin{equation}}
\newcommand{\ee}{\end{equation}}
\newcommand{\lb }{\left<}
\newcommand{\rb }{\right>}
\renewcommand{\l }{\left}
\renewcommand{\r }{\right}
\renewcommand{\O}[1]{\mathcal{O}\l(#1\r)}
\newcommand{\vtha}{v_{tha}}
\DeclarePairedDelimiter\ceil{\lceil}{\rceil}
\DeclarePairedDelimiter\floor{\lfloor}{\rfloor}
\newcommand{\icol}[1]{% inline column vector
  \left(\begin{smallmatrix}#1\end{smallmatrix}\right)}
\newcommand{\irow}[1]{% inline row vector
  \begin{smallmatrix}(#1)\end{smallmatrix}}
\def\be{\begin{equation}}
\def\ee{\end{equation}}
\def\beq{\begin{eqnarray}}
\def\eeq{\end{eqnarray}}
\let\oldhat\hat
\renewcommand{\v}[1]{\mathbf{#1}}
\renewcommand{\l }{\left}
%\renewcommand{\r }{\right}
\renewcommand{\O}[1]{\mathcal{O}\l(#1\r)}

\usepackage[normalem]{ulem}
\newcommand\redsout{\bgroup\markoverwith{\textcolor{red}{\rule[0.5ex]{2pt}{0.4pt}}}\ULon}

\PassOptionsToPackage{unicode}{hyperref}
\PassOptionsToPackage{naturalnames}{hyperref}

%% Bibliography commands from JPP Style
\RequirePackage{etoolbox}
\AtEndPreamble{%
\RequirePackage{natbib}
}
\newcommand\bibls{\kern.065em\relax}
\let\jppthebib\thebibliography
\let\jppendthebib\endthebibliography



% Author : Baptiste J. Frei
% 2018, EPFL

\renewcommand{\vec}[1]{\bm{#1}}
\newcommand{\E}{\vec{E}}
\newcommand{\B}{\vec{B}}
\renewcommand{\j}{J}
\renewcommand{\r}{\vec{r}}
\newcommand{\x}{\bm{x}}
\newcommand{\vi}{\bm{v}}
\newcommand{\y}{\bm{y}}
\newcommand{\z}{\bm{z}}
\newcommand{\A}{\bm{A}}
\newcommand{\e}{\bm{e}}
\newcommand{\R}{\bm{R}}
\newcommand{\U}{\bm{U}}
\renewcommand{\e}{\bm{e}}
\renewcommand{\c}{\bm{c}}
\renewcommand{\a}{\bm{a}}
\renewcommand{\b}{\vec{b}}
\newcommand{\bgamma}{\bar{\gamma}}
\newcommand{\rhoa}{\bm{\rho}_a}
\renewcommand{\u}{\vec{u}}
\newcommand{\cperp}{\c'_{\perp}}
\newcommand{\g}{\bm{g}}
\newcommand{\Z}{\bm{Z}} 


% Mathematical operators 
\newcommand{\grad}{\nabla}
\newcommand{\ptheta}{\frac{ \partial }{\partial \theta}}
\newcommand{\pmu}{\frac{ \partial }{\partial \mu}}
\newcommand{\pvparallel}{\frac{\partial }{ \partial v_{\parallel}}}
\renewcommand{\L}{\mathcal{L}}

% Gyrocentre operators
\newcommand{\gyptheta}{\frac{ \partial }{\partial \overline{\theta}}}
\newcommand{\gyppparallel}{\frac{ \partial }{\partial \overline{p}_{\parallel}}}
\newcommand{\gypvparallel}{\frac{ \partial }{\partial \overline{v}_{\parallel}}}
\newcommand{\gypmu}{\frac{ \partial }{\partial \overline{\mu}}}
\newcommand{\gygrad}{\overline{\grad}}
\newcommand{\curvature}{\bm{\kappa}}

% Special functions
\newcommand{\kernel}[1]{\mathcal{K}_{#1}}
\newcommand{\norm}[1]{\left\lVert#1\right\rVert}
\newcommand{\gyaver}[1]{\left<  #1 \right>}


% Coordinates and variables
\newcommand{\sparallel}{\overline{s}_{\parallel a}}
\newcommand{\sperp}{\overline{s}_{\perp a}}
\newcommand{\uparallel}{\overline{u}_{\parallel a}}
\newcommand{\vthparallel}{\overline{v}_{th\parallel a }}
\newcommand{\vthperp}{\overline{v}_{th \perp a}}
\newcommand{\Tperp}{\overline{T}_{\perp a}}
\newcommand{\Tparallel}{\overline{T}_{ \parallel a }}
\newcommand{\Pperp}{\overline{P}_{\perp a}}
\newcommand{\Pparallel}{\overline{P}_{ \parallel a }}
\newcommand{\gyrhoa}{\overline{\bm{\rho}}_a}
\newcommand{\vparallel}{v_{\parallel}}
\newcommand{\vperp}{v_{\perp}}
\newcommand{\vth}{v_{tha}}
\newcommand{\Qparallel}{Q_{\parallel  a}}
\newcommand{\Qperp}{Q_{\perp a}}
\newcommand{\gyU}{\overline{\U}}
\newcommand{\gyuparallel}{\overline{u}_{\parallel a}}
\newcommand{\gyvthparallel}{\overline{v}_{th \parallel a}}
\newcommand{\gyTperp}{\overline{T}_{\perp a}}
\newcommand{\gyTparallel}{\overline{T}_{\parallel a}}
\newcommand{\gyR}{\overline{\R}}
\newcommand{\gymu}{\overline{\mu}}
\newcommand{\gytheta}{\overline{\theta}}
\newcommand{\gypparallel}{\overline{p}_{\parallel}}
\newcommand{\gyvparallel}{\overline{v}_{\parallel}}
\newcommand{\gyvi}{\overline{\vi}}
\newcommand{\gyvperp}{\overline{v}_\perp}
\newcommand{\gyZ}{ \overline{\Z}}


% Hermite-Laguerre projector
\newcommand{\momentstar}[2]{ \norm{#2}^{*#1}_a}
\newcommand{\ovmomentstar}[2]{ \overline{\norm{#2}}^{*#1}_a}
\newcommand{\moment}[2]{ \norm{#2}^{#1}_a}
\newcommand{\momenta}[1]{\norm{#1}_a}
\newcommand{\momentastar}[1]{\norm{#1}^*_a}
\newcommand{\phaseV}{\mathcal{V}}
\newcommand{\phaseM}{\mathcal{M}}
\newcommand{\T}{\mathcal{T}}         

% Gyro-center densities
\newcommand{\gyN}{N_a}

% Distribution functions 
\newcommand{\gyFa}{\overline{F}_a}   
\newcommand{\gyFe}{\overline{F}_e} 
\newcommand{\gyFi}{\overline{F}_i} 

% Generating functions
\newcommand{\gR}{\bm{g}^{\R}}
\newcommand{\gtheta}{g^{\theta}}
\newcommand{\gmu}{g^{\mu}}
\newcommand{\gparallel}{g^{\parallel}}
\renewcommand{\gg}{\overline{g}}
\newcommand{\ggR}{\overline{\bm{g}}^{\R}}
\newcommand{\ggtheta}{\overline{g}^{\theta}}
\newcommand{\ggmu}{\overline{g}^{\mu}}
\newcommand{\ggparallel}{\overline{g}^{\parallel}}